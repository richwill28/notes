\documentclass[11pt,twoside=off,numbers=noenddot]{scrbook}

%%fakesection Load packages

\usepackage{lmodern}
\usepackage[pdfusetitle]{hyperref}
\ExplSyntaxOn
\sys_if_engine_luatex:T {
    \usepackage{luatex85}
}
\sys_if_engine_pdftex:T {
    \usepackage[T1]{fontenc}
}
\ExplSyntaxOff

% These are evan.sty
\usepackage{amsmath,amssymb,amsthm}
\usepackage{mathrsfs}
\usepackage[usenames,svgnames,dvipsnames]{xcolor}
\usepackage{textcomp}
\usepackage{enumerate}
\usepackage[textsize=scriptsize,shadow]{todonotes}
\usepackage{mathtools}
\usepackage{microtype}
\usepackage[normalem]{ulem}
\usepackage{stmaryrd}
\usepackage{wasysym}
\usepackage{multirow}
\usepackage{prerex}
\usepackage[nameinlink]{cleveref}

%%fakesection evan.sty macros
%Small commands
%% Napkin commands
\newcommand{\prototype}[1]{
    \emph{{\color{red} Prototypical example for this section:} #1} \par\medskip
}
\newenvironment{moral}{%
    \begin{mdframed}[linecolor=green!70!black]%
        \bfseries\color{green!50!black}}%
        {\end{mdframed}}

%%fakesection Links (hyperref loaded earlier implicitly)
\hypersetup{
    linkcolor={red!50!black},
    citecolor={green!50!black},
    urlcolor={blue!80!black},
    pdfkeywords={napkin,math},
    pdfsubject={web.evanchen.cc},
    colorlinks,
}

%%fakesection Commutative diagrams
\usepackage{tikz-cd}
\usetikzlibrary{arrows,arrows.meta}
% make a larger hook
% https://tex.stackexchange.com/questions/514451/how-to-define-a-new-hooked-arrow
\makeatletter
\pgfdeclarearrow{
    name=xGlyph,
    cache=false,
    bending mode=none,
    parameters={\tikzcd@glyph@len,\tikzcd@glyph@shorten},
    setup code={%
            \pgfarrowssettipend{\tikzcd@glyph@len\advance\pgf@x by\tikzcd@glyph@shorten}},
    defaults={
            glyph axis=axis_height,
            glyph length=+1.55ex,
            glyph shorten=+-0.1ex},
    drawing code={%
            \pgfpathrectangle{\pgfpoint{+0pt}{+-1.5ex}}{\pgfpoint{+\tikzcd@glyph@len}{+3ex}}%
            \pgfusepathqclip%
            \pgftransformxshift{+\tikzcd@glyph@len}%
            \pgftransformyshift{+-\tikzcd@glyph@axis}%
            \pgftext[right,base]{\tikzcd@glyph}}}
\makeatother
\tikzcdset{
arrow style=tikz,
diagrams={>={Latex}},
tikzcd left hook/.tip={xGlyph[glyph math command=supset, swap, glyph axis = 5.7pt]},
tikzcd right hook/.tip={xGlyph[glyph math command=supset, glyph axis = 5.7pt]},
surjective head arrow /.tip = {tikzcd to[sep=-1.5pt]tikzcd to},
surjective head/.style={
        -surjective head arrow
    }
}

%%fakesection Page layout
\usepackage[headsepline]{scrlayer-scrpage}
\renewcommand{\headfont}{}
\addtolength{\textheight}{3.14cm}
\setlength{\footskip}{0.5in}
\setlength{\headsep}{10pt}

\def\shortdate{\leavevmode\hbox{\the\year-\twodigits\month-\twodigits\day}}
\def\twodigits#1{\ifnum#1<10 0\fi\the#1}
\automark[chapter]{chapter}

\rohead{\footnotesize\thepage}
\rehead{\footnotesize \textbf{\sffamily Napkin}, by \emph{Evan Chen} (\napkinversion)}
\lehead{\footnotesize\thepage}
\lohead{\footnotesize \leftmark}
\chead{}
\rofoot{}
\refoot{}
\lefoot{}
\lofoot{}
%\cfoot{\pagemark}

%%fakesection Fancy section and chapter heads
\renewcommand*{\sectionformat}{\color{purple}\S\thesection\autodot\enskip}
\renewcommand*{\subsectionformat}{\color{purple}\S\thesubsection\autodot\enskip}
\newcommand{\problemhead}{A few harder problems to think about}
\renewcommand{\thesubsection}{\thesection.\roman{subsection}}

\addtokomafont{chapterprefix}{\raggedleft}
\RedeclareSectionCommand[beforeskip=0.5em]{chapter}
\renewcommand*{\chapterformat}{%
    \mbox{\scalebox{1.5}{\chapappifchapterprefix{\nobreakspace}}%
        \scalebox{2.718}{\color{purple}\thechapter\autodot}\enskip}}

\addtokomafont{partprefix}{\rmfamily}
\renewcommand*{\partformat}{\color{purple}\scalebox{2.5}{\thepart}}

%%fakesection Theorems
\usepackage{thmtools}
\usepackage[framemethod=TikZ]{mdframed}

\theoremstyle{definition}
\mdfdefinestyle{mdbluebox}{%
    linewidth=1pt,
    skipabove=12pt,
    innerbottommargin=9pt,
    skipbelow=2pt,
    nobreak=true,
    linecolor=blue,
    backgroundcolor=TealBlue!5,
}
\declaretheoremstyle[
    headfont=\sffamily\bfseries\color{MidnightBlue},
    mdframed={style=mdbluebox},
    headpunct={\\[3pt]},
    postheadspace={0pt}
]{thmbluebox}

\mdfdefinestyle{mdredbox}{%
    linewidth=0.5pt,
    skipabove=12pt,
    frametitleaboveskip=5pt,
    frametitlebelowskip=0pt,
    skipbelow=2pt,
    frametitlefont=\bfseries,
    innertopmargin=4pt,
    innerbottommargin=8pt,
    linecolor=RawSienna,
    backgroundcolor=Salmon!5,
}
\declaretheoremstyle[
    headfont=\bfseries\color{RawSienna},
    mdframed={style=mdredbox},
    headpunct={\\[3pt]},
    postheadspace={0pt},
]{thmredbox}

\declaretheorem[%
    style=thmbluebox,name=Theorem,numberwithin=section]{theorem}
\declaretheorem[style=thmbluebox,name=Lemma,sibling=theorem]{lemma}
\declaretheorem[style=thmbluebox,name=Proposition,sibling=theorem]{proposition}
\declaretheorem[style=thmbluebox,name=Corollary,sibling=theorem]{corollary}
\declaretheorem[style=thmredbox,name=Example,sibling=theorem]{example}

\mdfdefinestyle{mdgreenbox}{%
    skipabove=8pt,
    linewidth=2pt,
    rightline=false,
    leftline=true,
    topline=false,
    bottomline=false,
    linecolor=ForestGreen,
    backgroundcolor=ForestGreen!5,
}
\declaretheoremstyle[
    headfont=\bfseries\sffamily\color{ForestGreen!70!black},
    bodyfont=\normalfont,
    spaceabove=2pt,
    spacebelow=1pt,
    mdframed={style=mdgreenbox},
    headpunct={ --- },
]{thmgreenbox}
\declaretheoremstyle[
    headfont=\bfseries\sffamily\color{ForestGreen!70!black},
    bodyfont=\normalfont,
    spaceabove=2pt,
    spacebelow=1pt,
    mdframed={style=mdgreenbox},
    headpunct={},
]{thmgreenbox*}

\mdfdefinestyle{mdblackbox}{%
    skipabove=8pt,
    linewidth=3pt,
    rightline=false,
    leftline=true,
    topline=false,
    bottomline=false,
    linecolor=black,
    backgroundcolor=RedViolet!5!gray!5,
}
\declaretheoremstyle[
    headfont=\bfseries,
    bodyfont=\normalfont\small,
    spaceabove=0pt,
    spacebelow=0pt,
    mdframed={style=mdblackbox}
]{thmblackbox}

\declaretheorem[name=Question,sibling=theorem,style=thmblackbox]{ques}
\declaretheorem[name=Exercise,sibling=theorem,style=thmblackbox]{exercise}
\declaretheorem[name=Remark,sibling=theorem,style=thmgreenbox]{remark}
\declaretheorem[name=Remark,sibling=theorem,style=thmgreenbox*]{remark*}
\declaretheorem[name=Step,style=thmgreenbox]{step} % only used in Lebesgue int

\definecolor{darkmagenta}{rgb}{0.55, 0.0, 0.55}
\definecolor{patriarch}{rgb}{0.5, 0.0, 0.5}

\theoremstyle{definition}
\mdfdefinestyle{mdpurplebox}{%
    linewidth=1pt,
    skipabove=12pt,
    innerbottommargin=9pt,
    skipbelow=2pt,
    nobreak=true,
    linecolor=darkmagenta,
    backgroundcolor=patriarch!5,
}
\declaretheoremstyle[
    headfont=\sffamily\bfseries\color{darkmagenta},
    mdframed={style=mdpurplebox},
    headpunct={\\[3pt]},
    postheadspace={0pt}
]{thmpurplebox}
\declaretheorem[style=thmpurplebox,name=Definition,numberwithin=section]{definition}
\newtheorem{claim}[theorem]{Claim}
\newtheorem{fact}[theorem]{Fact}
\newtheorem{abuse}[theorem]{Abuse of Notation}

\newtheorem{problem}{Problem}[chapter]
\renewcommand{\theproblem}{\thechapter\Alph{problem}}
\newtheorem{sproblem}[problem]{Problem}
\newtheorem{dproblem}[problem]{Problem}
\renewcommand{\thesproblem}{\theproblem$^{\star}$}
\renewcommand{\thedproblem}{\theproblem$^{\dagger}$}
\newcommand{\listhack}{$\empty$\vspace{-2em}}

%%fakesection Answers
\usepackage{answers}
\Newassociation{hint}{answeritem}{tex/backmatter/all-hints}
\Newassociation{sol}{answeritem}{tex/backmatter/all-solns}
\renewcommand{\solutionextension}{out}
\renewenvironment{answeritem}[1]{\item[\bfseries #1.]}{}

%%fakesection Table of contents
% First add ToC to ToC
\makeatletter
\usepackage{etoolbox}
\pretocmd{\tableofcontents}{%
    \if@openright\cleardoublepage\else\clearpage\fi
    \pdfbookmark[0]{\contentsname}{toc}%
}{}{}%
\makeatother
\setcounter{tocdepth}{1}
\RedeclareSectionCommand[tocnumwidth=4.2em]{part}
\RedeclareSectionCommand[tocpagenumberwidth=2.2em,tocnumwidth=4.2em]{chapter}
\RedeclareSectionCommand[tocpagenumberwidth=2.2em,tocnumwidth=2.8em]{section}
% adjust tocpagenumberwidth manually for large page number: https://tex.stackexchange.com/a/502168

%%fakesection Asymptote definitions
\usepackage{patch-asy}
\numberwithin{asy}{chapter}
\renewcommand{\theasy}{\thechapter\Alph{asy}}
\begin{asydef}
    import extras;
    size(6cm);
    usepackage("amsmath");
    usepackage("amssymb");
    defaultpen(fontsize(11pt));
    settings.tex = "latex";
    settings.outformat = "pdf";
\end{asydef}
\def\asydir{asy}

%%fakesection Bibliography
\usepackage[backend=biber,backref=true,style=alphabetic]{biblatex}
\DeclareLabelalphaTemplate{
    \labelelement{
        \field[final]{shorthand}
        \field{label}
        \field[strwidth=2,strside=left]{labelname}
    }
    \labelelement{
        \field[strwidth=2,strside=right]{year}
    }
}
\DeclareFieldFormat{labelalpha}{\textbf{\scriptsize #1}}
\addbibresource{references.bib}
\addbibresource{images.bib}
%% stylistic biblatex choices
\DefineBibliographyStrings{english}{%
    backrefpage  = {cited p.}, % for single page number
    backrefpages = {cited pp.} % for multiple page numbers
}
\DeclareFieldFormat{journaltitle}{\mkbibemph{#1},} % italic journal title with comma
\DeclareFieldFormat[inbook,thesis]{title}{\mkbibemph{#1}\addperiod} % italic title with period
\DeclareFieldFormat[article]{title}{#1} % title of journal article is printed as normal text
\DeclareFieldFormat[article]{volume}{\textbf{#1}\addcolon\space}
\renewcommand{\mkbibnamegiven}[1]{\textsc{#1}}
\renewcommand{\mkbibnamefamily}[1]{\textsc{#1}}
\renewcommand{\mkbibnameprefix}[1]{\textsc{#1}}
\renewcommand{\mkbibnamesuffix}[1]{\textsc{#1}}
\renewcommand{\finentrypunct}{}

%%fakesection Mini ToC
\usepackage[tight]{minitoc}
\mtcsetfont{parttoc}{chapter}{\sffamily\bfseries}
\mtcsetfont{parttoc}{section}{\footnotesize\rmfamily\upshape\mdseries}
\mtcsetfont{parttoc}{subsection}{\footnotesize\rmfamily\upshape\mdseries}
%\mtcsetdepth{parttoc}{1}
\setcounter{parttocdepth}{1}
\renewcommand*{\partheadstartvskip}{\vspace*{20em}}
\renewcommand*{\partheadendvskip}{}
%\noptcrule
\renewcommand\beforeparttoc{\noindent{\bfseries \Large Part \thepart: Contents}}
%\hspace{\fill}\rule{0.95\linewidth}{2pt}\hspace{\fill}
\doparttoc[n]

%%fakesection Misc haxx
\pdfstringdefDisableCommands{\def\Spec{\text{Spec }}\def\sigma{σ}}
\usepackage{parskip}

% Style
\newenvironment{proofidea}{\begin{proof}[Proof Idea]}{\end{proof}}
\newcommand{\vocab}[1]{\textbf{\color{blue} #1}}

% Computer Science
\renewcommand{\L}{\textbf{L}}
\newcommand{\coNL}{\textbf{coNL}}
\newcommand{\NL}{\textbf{NL}}
\newcommand{\NSPACE}{\textbf{NSPACE}}
\newcommand{\PSPACE}{\textbf{PSPACE}}
\newcommand{\SPACE}{\textbf{SPACE}}
\newcommand{\TQBF}{\textbf{TQBF}}

% Mathematics
\newcommand{\CC}{\mathbb C}
\newcommand{\FF}{\mathbb F}
\newcommand{\NN}{\mathbb N}
\newcommand{\QQ}{\mathbb Q}
\newcommand{\RR}{\mathbb R}
\newcommand{\ZZ}{\mathbb Z}


\title{Notes on Programming Language Theory}
\author{Richard Willie}

\begin{document}

\maketitle

\chapter*{Preface}
These notes are a loose amalgamation of ideas, concepts, and
explanations drawn from various books, papers, and personal
reflections. They were originally meant for my own understanding and
organization of thoughts, and as such, they may be unpolished,
incomplete, or even occasionally incorrect.

I share them in the hope that they may serve as a useful reference,
but they should not be treated as a primary source of learning.
Readers are strongly encouraged to consult original texts and
authoritative resources for a more rigorous and accurate treatment of
the topics discussed.

Use these notes as a companion to your studies, not as a substitute
for the depth and clarity provided by well-established literature.

\tableofcontents

\newpage

\chapter{Typed Arithmetic Expressions}
\chplabel{typed-arithmetic-expressions}

\section{Types}
The syntax for arithmetic expressions:
\begin{align*}
  \ttt{t} ::= & \quad && \text{term} \\
  & \quad \ttt{true} && \text{constant true} \\
  & \quad \ttt{false} && \text{constant false} \\
  & \quad \ttt{if t then t else t} && \text{conditional} \\
  & \quad \ttt{0} && \text{constant zero} \\
  & \quad \ttt{succ t} && \text{successor} \\
  & \quad \ttt{pred t} && \text{predecessor} \\
  & \quad \ttt{iszero t} && \text{zero test}
\end{align*}

Evaluating a term can either result in a value...
\begin{align*}
  \ttt{v} ::= & \quad && \text{value} \\
  & \quad \ttt{true} && \text{true value} \\
  & \quad \ttt{false} && \text{false value} \\
  & \quad \ttt{nv} && \text{numeric value} \\
  \\
  \ttt{nv} ::= & \quad && \text{numeric value} \\
  & \quad \ttt{0} && \text{zero value} \\
  & \quad \ttt{succ nv} && \text{successor value}
\end{align*}
or else get \textit{stuck} at some stage, by reaching a term for
which no evaluation rule applies.

Stuck terms correspond to meaningless or erroneous programs. We would
therefore like to be able to tell, without actually evaluating a
term, that its evaluation will definitely \textit{not} get stuck. To
do this, we introduce two types, $\ttt{Nat}$ and $\ttt{Bool}$, to
distinguish between terms whose result will be a numeric value and
terms whose result will be a boolean.

\section{The Typing Relation}
The typing relation for arithmetic expressions, written $\texttt{t} :
\texttt{T}$, is defined by a set of inference rules assigning types
to terms, summarized as follows.

New syntactic forms:
\begin{align*}
  \texttt{T} ::= & \quad && \text{type} \\
  & \quad \texttt{Bool} && \text{type of booleans} \\
  & \quad \texttt{Nat} && \text{type of natural numbers}
\end{align*}

New typing rules:
\begin{prooftree}
  \AxiomC{$\ttt{true} : \ttt{Bool}$ (\tsc{T-True})}
\end{prooftree}

\begin{prooftree}
  \AxiomC{$\ttt{false} : \ttt{Bool}$ (\tsc{T-False})}
\end{prooftree}

\begin{prooftree}
  \AxiomC{$\ttt{t}_1 : \ttt{Bool}$}
  \AxiomC{$\ttt{t}_2 : \ttt{T}$}
  \AxiomC{$\ttt{t}_3 : \ttt{T}$}
  \RightLabel{(\tsc{T-If})}
  \TrinaryInfC{$\ttt{if $\ttt{t}_1$ then $\ttt{t}_2$ else
  $\ttt{t}_3$} : \ttt{T}$}
\end{prooftree}

\begin{prooftree}
  \AxiomC{$\ttt{0} : \ttt{Nat}$ (\tsc{T-Zero})}
\end{prooftree}

\begin{prooftree}
  \AxiomC{$\ttt{t}_1 : \ttt{Nat}$}
  \RightLabel{(\tsc{T-Succ})}
  \UnaryInfC{$\ttt{succ $\ttt{t}_1$} : \ttt{Nat}$}
\end{prooftree}

\begin{prooftree}
  \AxiomC{$\ttt{t}_1 : \ttt{Nat}$}
  \RightLabel{(\tsc{T-Pred})}
  \UnaryInfC{$\ttt{pred $\ttt{t}_1$} : \ttt{Nat}$}
\end{prooftree}

\begin{prooftree}
  \AxiomC{$\ttt{t}_1 : \ttt{Nat}$}
  \RightLabel{(\tsc{T-IsZero})}
  \UnaryInfC{$\ttt{iszero $\ttt{t}_1$} : \ttt{Bool}$}
\end{prooftree}

\begin{definition}
  Formally, the \vocab{typing relation} for arithmetic expressions is
  the smallest binary relation between terms and types satisfying all
  instances of the rules above. A term $\ttt{t}$ is \vocab{well
  typed} if there is some $\ttt{T}$ such that $\ttt{t} : \ttt{T}$.
\end{definition}

When reasoning about the typing relation, we will often make
statements like ``If a term of the form $\ttt{succ\,$\ttt{t}_1$}$ has
any type at all, then it has type $\ttt{Nat}$.'' The following lemma
gives us a compendium of basic statements of this form, each
following immediately from the shape of the corresponding typing rule.

\begin{lemma}[Inversion lemma for typed arithmetic expressions]
  1. If $\ttt{true} : \ttt{R}$, then $\ttt{R} = \ttt{Bool}$.

  2. If $\ttt{false} : \ttt{R}$, then $\ttt{R} = \ttt{Bool}$.

  3. If $\ttt{if $\ttt{t}_1$ then $\ttt{t}_2$ else $\ttt{t}_3$} :
  \ttt{R}$, then $\ttt{t}_1 : \ttt{Bool}$ and $\ttt{t}_2 : \ttt{R}$
  and $\ttt{t}_3 : \ttt{R}$.

  4. If $\ttt{0} : \ttt{R}$, then $\ttt{R} = \ttt{Nat}$.

  5. If $\ttt{succ $\ttt{t}_1$} : \ttt{R}$, then $\ttt{t}_1 :
  \ttt{Nat}$ and $\ttt{R} = \ttt{Nat}$.

  6. If $\ttt{pred $\ttt{t}_1$} : \ttt{R}$, then $\ttt{t}_1 :
  \ttt{Nat}$ and $\ttt{R} = \ttt{Nat}$.

  7. If $\ttt{iszero $\ttt{t}_1$} : \ttt{R}$, then $\ttt{t}_1 :
  \ttt{Nat}$ and $\ttt{R} = \ttt{Bool}$.
\end{lemma}

\begin{proof}
  Immediate from the definition of the typing relation.
\end{proof}

The inversion lemma is sometimes called the \textit{generation lemma}
for the typing relation, since, given a valid typing statement, it
shows how a proof of this statement could have been generated. The
inversion lemma leads directly to a recursive algorithm for
calculating the types of terms, since it tells us, for a term of each
syntactic form, how to calculate its type (if it has one) from the
types of its subterms.

\begin{theorem}[Uniqueness of types for typed arithmetic expressions]
  Each term $\ttt{t}$ has at most one type. That is, if $\ttt{t}$ is
  typable, then its type is unique. Moreover, there is just one
  derivation of this typing built from the inference rules above.
\end{theorem}

\begin{proof}
  Straightforward structural induction on $\ttt{t}$, using the
  appropriate clause of the inversion lemma (plus the induction
  hypothesis) for each case.
\end{proof}

\section{Safety = Progress + Preservation}
The most basic property of this type system or any other is
\textit{safety} (or \textit{soundness}), i.e. well-typed terms do not
go wrong. We have already chosen how to formalize what it means for a
term to go wrong: it means reaching a stuck state that is not
designated as a final value but where evaluation rules do not tell us
what do next. What we want to know, then, is that well-typed terms do
not get stuck. We show this in two steps, commonly known as the
\textit{progress} and \textit{preservation} theorems.

\begin{theorem}[Progress]
  \thmlabel{progress-for-typed-arithmetic-expressions}
  A well-typed term is not stuck (either it is a value or it can take
  a step according to the evaluation rules). Formally, suppose
  $\ttt{t}$ is a well-typed term (that is, $\ttt{t} : \ttt{T}$ for
  some $\ttt{T}$). Then either $\ttt{t}$ is a value, or else there is
  some $\ttt{t}'$ with $\ttt{t} \rightarrow \ttt{t}'$.
\end{theorem}

\begin{proof}
  By induction on a derivation of $\ttt{t} : \ttt{T}$. The
  \tsc{T-True}, \tsc{T-False}, and \tsc{T-Zero} cases are immediate,
  since $\ttt{t}$ in these cases is a value. For the other caes, we
  argue as follows.

  Case \tsc{T-If}: $\ttt{t} = \ttt{if $\ttt{t}_1$ then $\ttt{t}_2$ else
  $\ttt{t}_3$}$, where $\ttt{t}_1 : \ttt{Bool}$, $\ttt{t}_2 :
  \ttt{T}$, and $\ttt{t}_3 : \ttt{T}$.
  By the induction hypothesis, either $\ttt{t}_1$ is a value or else
  there is some $\ttt{t}_1'$ such that $\ttt{t}_1 \rightarrow
  \ttt{t}_1'$. If $\ttt{t}_1$ is a value, then it must be either
  $\ttt{true}$ or $\ttt{false}$, in which case either \tsc{E-IfTrue}
  or \tsc{E-IfFalse} applies to $\ttt{t}$. On the other hand, if
  $\ttt{t}_1 \rightarrow \ttt{t}_1'$, then by \tsc{E-If}, we have
  $\ttt{t} \rightarrow \ttt{if $\ttt{t}_1'$ then $\ttt{t}_2$ else $\ttt{t}_3$}$.

  Case \tsc{T-Succ}: $\ttt{t} = \ttt{succ $\ttt{t}_1$}$, where
  $\ttt{t}_1 : \ttt{Nat}$. By the induction hypothesis, either
  $\ttt{t}_1$ is a value or else there is some $\ttt{t}_1'$ such that
  $\ttt{t}_1 \rightarrow \ttt{t}_1'$. If $\ttt{t}_1$ is a value, then
  it must be a numeric value, in which case so is $\ttt{t}$. On the
  other hand, if $\ttt{t}_1 \rightarrow \ttt{t}_1'$, then by
  \tsc{E-Succ}, we have $\ttt{t} \rightarrow \ttt{succ $\ttt{t}_1'$}$.

  Case \tsc{T-Pred}: $\ttt{t} = \ttt{pred $\ttt{t}_1$}$, where
  $\ttt{t}_1 : \ttt{Nat}$. By the induction hypothesis, either
  $\ttt{t}_1$ is a value or else there is some $\ttt{t}_1'$ such that
  $\ttt{t}_1 \rightarrow \ttt{t}_1'$. If $\ttt{t}_1$ is a value, then
  it must be a numeric value, either $\ttt{0}$ or $\ttt{succ
  $\ttt{nv}_1$}$ for some $\ttt{nv}_1$, in which case either
  \tsc{E-PredZero} or \tsc{E-PredSucc} applies to $\ttt{t}$. On the
  other hand, if $\ttt{t}_1 \rightarrow \ttt{t}_1'$, then by
  \tsc{E-Pred}, we have $\ttt{t} \rightarrow \ttt{pred $\ttt{t}_1'$}$.

  Case \tsc{T-IsZero}: $\ttt{t} = \ttt{iszero $\ttt{t}_1$}$, where
  $\ttt{t}_1 : \ttt{Nat}$. Similar.
\end{proof}

\begin{theorem}[Preservation]
  If a well-typed term takes a step of evaluation, then the resulting
  term is also well typed. Formally, if $\ttt{t} : \ttt{T}$ and
  $\ttt{t} \rightarrow \ttt{t}'$, then $\ttt{t}' : \ttt{T}$.
\end{theorem}

\begin{proof}
  By induction on a derivation of $\ttt{t} : \ttt{T}$. At each step
  of the induction, we assume that the desired property holds for all
  subderivations and proceed by case analysis on the final rule of
  the derivation.

  Case \tsc{T-True}: $\ttt{t} = \ttt{true}$ and $\ttt{T} =
  \ttt{Bool}$. Vacuously true.

  Case \tsc{T-False}: $\ttt{t} = \ttt{false}$ and $\ttt{T} =
  \ttt{Bool}$. Vacuously true.

  Case \tsc{T-If}: $\ttt{t} = \ttt{if $\ttt{t}_1$ then $\ttt{t}_2$ else
  $\ttt{t}_3$}$, where $\ttt{t}_1 : \ttt{Bool}$,
  $\ttt{t}_2 : \ttt{T}$, and $\ttt{t}_3 : \ttt{T}$. There are three
  rules by which $\ttt{t} \rightarrow \ttt{t}'$ can be derived: \tsc{E-IfTrue},
  \tsc{E-IfFalse}, and \tsc{E-If}. We consider each case separately
  (omitting \tsc{E-IfFalse}).
  \begin{itemize}
    \item Subcase \tsc{E-IfTrue}: $\ttt{t}_1 = \ttt{true}$ and
      $\ttt{t}' = \ttt{t}_2$. If $\ttt{t} \rightarrow \ttt{t}'$ is
      derived using \tsc{E-IfTrue}, then from the form of this rule
      we see that $\ttt{t}_1$ must be $\ttt{true}$ and the resulting
      term $\ttt{t}'$ is the second subexpression $\ttt{t}_2$. This
      means we are done, since we know (by the assumption of the
      \tsc{T-If} case) that $\ttt{t}_2 : \ttt{T}$, which is what we need.
    \item Subcase \tsc{E-If}: $\ttt{t}_1 \rightarrow
      \ttt{t}_1'$ and $\ttt{t}' = \ttt{if $\ttt{t}_1'$ then
      $\ttt{t}_2$ else $\ttt{t}_3$}$. From the assumptions of the
      \tsc{T-If} case, we have a subderivation of the original typing
      derivation whose conclusion is $\ttt{t}_1 : \ttt{Bool}$. We can
      apply the induction hypothesis to this subderivation, obtaining
      $\ttt{t}_1' : \ttt{Bool}$. Combining this with the facts (from
      the assumptions of the \tsc{T-If} case) that $\ttt{t}_2 :
      \ttt{T}$ and $\ttt{t}_3 : \ttt{T}$, we can apply rule
      \tsc{T-If} to conclude that $\ttt{if $\ttt{t}_1'$ then
      $\ttt{t}_2$ else $\ttt{t}_3$} : \ttt{T}$.
  \end{itemize}

  Case \tsc{T-Zero}: $\ttt{t} = \ttt{0}$ and $\ttt{T} = \ttt{Nat}$.
  Vacuously true.

  Case \tsc{T-Succ}: $\ttt{t} = \ttt{succ $\ttt{t}_1$}$, where
  $\ttt{T} = \ttt{Nat}$ and $\ttt{t}_1 : \ttt{Nat}$. There is just
  one rule, \tsc{E-Succ}, that can be used to derive $\ttt{t}
  \rightarrow \ttt{t}'$. The form of this rule tells us that
  $\ttt{t}_1 \rightarrow \ttt{t}_1'$. Since we also know $\ttt{t}_1 :
  \ttt{Nat}$, we can apply the induction hypothesis to obtain
  $\ttt{t}_1' : \ttt{Nat}$, from which we obtain $\ttt{succ
  $\ttt{t}_1'$} : \ttt{Nat}$ by applying rule \tsc{T-Succ}.

  Case \tsc{T-Pred}: $\ttt{t} = \ttt{pred $\ttt{t}_1$}$, where
  $\ttt{T} = \ttt{Nat}$ and $\ttt{t}_1 : \ttt{Nat}$. Similar.

  Case \tsc{T-IsZero}: $\ttt{t} = \ttt{iszero $\ttt{t}_1$}$, where
  $\ttt{T} = \ttt{Bool}$ and $\ttt{t}_1 : \ttt{Nat}$. Similar.
\end{proof}

\chapter{Simply Typed Lambda Calculus}
\chplabel{simply-typed-lambda-calculus}

\section{The Typing Relation}
\seclabel{simply-typed-lambda-calculus-the-typing-relation}
Since terms may contain nested $\lambda$-abstractions, we will need,
in general, to talk about several assumptions. This changes the
typing relation from a two-place relation, $\ttt{t} : \ttt{T}$, to a
three-place relation, $\Gamma \vdash \ttt{t} : \ttt{T}$, where
$\Gamma$ is a set of assumptions about the types of the free
variables in $\ttt{t}$.

\begin{definition}
  Formally, a \vocab{typing context} (or a \vocab{type environment})
  $\Gamma$ is a sequence of variables and their types, and the comma
  operator extends $\Gamma$ by adding a new binding on the right.
\end{definition}

The rule of typing abstraction has the general form:
\begin{prooftree}
  \AxiomC{$\Gamma, \ttt{x:$\ttt{T}_1$} \vdash \ttt{t}_2 : \ttt{T}_2
  \quad$} \RightLabel{(\tsc{T-Abs})}
  \UnaryInfC{$\Gamma \vdash \ttt{$\lambda$x:$\ttt{T}_1$.$\ttt{t}_2$}
  : \ttt{$\ttt{T}_1$$\rightarrow$$\ttt{T}_2$}$}
\end{prooftree}
where the premise adds one more assumption to those in the conclusion.

The typing rules for variables:
\begin{prooftree}
  \AxiomC{$\ttt{x:T} \in \Gamma$}
  \RightLabel{(\tsc{T-Var})}
  \UnaryInfC{$\Gamma \vdash \ttt{x} : \ttt{T}$}
\end{prooftree}

The typing rules for applications:
\begin{prooftree}
  \AxiomC{$\Gamma \vdash \ttt{t}_1 :
  \ttt{$\ttt{T}_{11}$$\rightarrow$$\ttt{T}_{12}$}$}
  \AxiomC{$\Gamma \vdash \ttt{t}_2 : \ttt{T}_{11}$}
  \RightLabel{(\tsc{T-App})}
  \BinaryInfC{$\Gamma \vdash \ttt{$\ttt{t}_1$\,$\ttt{t}_2$} : \ttt{T}_{12}$}
\end{prooftree}

For completeness, we summarize the syntax and evaluation rules for
simply-typed lambda calculus as follows.

Syntax:
\begin{align*}
  \ttt{t} ::= & \quad && \text{term} \\
  & \quad \ttt{x} && \text{variable} \\
  & \quad \ttt{$\lambda$x:T.t} && \text{abstraction} \\
  & \quad \ttt{t\,t} && \text{application} \\
  \\
  \ttt{v} ::= & \quad && \text{value} \\
  & \quad \ttt{$\lambda$x:T.t} && \text{abstraction value} \\
  \\
  \ttt{T} ::= & \quad && \text{type} \\
  & \quad \ttt{T$\rightarrow$T} && \text{type of functions}
\end{align*}

Evaluation:
\begin{prooftree}
  \AxiomC{$\ttt{t}_1 \rightarrow \ttt{t}_1'$}
  \RightLabel{(\tsc{E-App1})}
  \UnaryInfC{$\ttt{$\ttt{t}_1$\,$\ttt{t}_2$} \rightarrow
  \ttt{$\ttt{t}_1'$\,$\ttt{t}_2$}$}
\end{prooftree}

\begin{prooftree}
  \AxiomC{$\ttt{t}_2 \rightarrow \ttt{t}_2'$}
  \RightLabel{(\tsc{E-App2})}
  \UnaryInfC{$\ttt{$\ttt{v}_1$\,$\ttt{t}_2$} \rightarrow
  \ttt{$\ttt{v}_1$\,$\ttt{t}_2'$}$}
\end{prooftree}

\begin{prooftree}
  \AxiomC{$\ttt{($\lambda$x:$\ttt{T}_{11}$.$\ttt{t}_{2}$)\,$\ttt{v}_2$}
  \rightarrow \ttt{[x$\mapsto$$\ttt{v}_2$]}\ttt{t}_{2}$ (\tsc{E-AppAbs})}
\end{prooftree}

\section{Properties of Typing}
As in \chpref{typed-arithmetic-expressions}, we need to develop a few
basic lemmas before we can prove type safety. Most of these are
similar to what we saw before, we just need to add contexts to the
typing relation and add clauses to each proof for $\lambda$-
abstractions, applications, and variables. The only significant new
requirement is a substitution lemma for the typing relation
(\lemref{substitution-lemma-for-simply-typed-lambda-calculus}).

First off, an inversion lemma records a collection of observations
about how typing derivations are built: the clause for each syntactic
form tells us ``if a term of this form is well typed, then its
subterms must have types of these forms...''

\begin{lemma}[Inversion lemma for simply typed lambda calculus]
  \lemlabel{inversion-lemma-for-simply-typed-lambda-calculus}
  1. If $\Gamma \vdash \ttt{x} : \ttt{R}$, then $\ttt{x:R} \in \Gamma$.

  2. If $\Gamma \vdash \ttt{$\lambda$x:$\ttt{T}_1$.$\ttt{t}_2$} :
  \ttt{R}$, then $\ttt{R} =
  \ttt{$\ttt{T}_1$$\rightarrow$$\ttt{R}_2$}$ for some $\ttt{R}_2$
  with $\Gamma, \ttt{x:$\ttt{T}_1$} \vdash \ttt{t}_2 : \ttt{R}_2$.

  3. If $\Gamma \vdash \ttt{$\ttt{t}_1$\,$\ttt{t}_2$} : \ttt{R}$,
  then $\Gamma \vdash \ttt{t}_1 :
  \ttt{$\ttt{T}_{11}$$\rightarrow$$\ttt{R}$}$ and $\Gamma \vdash
  \ttt{t}_2 : \ttt{T}_{11}$ for some type $\ttt{T}_{11}$.
\end{lemma}

\begin{proof}
  Immediate from the definition of the typing relation.
\end{proof}

In \secref{simply-typed-lambda-calculus-the-typing-relation}, we
chose an explicitly typed presentation of the calculus to simplify
the job of typechecking. This involved adding type annotations to
bound variables in function abstractions, but nowhere else. In what
sense is this ``enough''? One answer is provided by the ``uniqueness
of types'' theorem, which tells us that well-typed terms are in
one-to-one correspondence with their typing derivations: the typing
derivation can be recovered uniquely from the term (and, of course,
vice versa). In fact, the correspondence is so straightforward that,
in a sense, there is little difference between the term and the derivation.

\begin{theorem}[Uniqueness of types for simply typed lambda calculus]
  \thmlabel{uniqueness-of-types-for-simply-typed-lambda-calculus}
  In a given typing context $\Gamma$, a term $\ttt{t}$ (with free
  variables all in the domain of $\Gamma$) has at most one type. That
  is, if a term is typable, then its type is unique. Moreover, there
  is just one derivation of this typing built from the inference
  rules that generate the typing relation.
\end{theorem}

\begin{proof}
  Straightforward.
\end{proof}

We can prove a progress theorem analogous to
\thmref{progress-for-typed-arithmetic-expressions}. The statement of
the theorem needs one small change: we are interested only in
\textit{closed} terms, with no free variables. For open terms, the
progress theorem actually fails: a term like $\ttt{x}$ is a normal
form, but not a value. However, this failure does not represent a
defect in the language, since complete programs, which are the terms
we actually care about evaluating, are always closed.

\begin{theorem}[Progress]
  Suppose $\ttt{t}$ is a closed, well-typed term (that is, $\vdash
  \ttt{t} : \ttt{T}$ for some $\ttt{T}$). Then either $\ttt{t}$ is a
  value or else there is some $\ttt{t}'$ with $\ttt{t} \rightarrow \ttt{t}'$.
\end{theorem}

\begin{proof}
  Straightforward induction on typing derivations. The variable case
  cannot occur (because $\ttt{t}$ is closed). The abstraction case is
  immediate, since abstractions are values. The only interesting case
  is the one for application, where $\ttt{t} = \ttt{$\ttt{t}_1$\,$\ttt{t}_2$}$
  with $\vdash \ttt{t}_1 :
  \ttt{$\ttt{T}_{11}\rightarrow\ttt{T}_{12}$}$ and $\vdash \ttt{t}_2
  : \ttt{T}_{11}$. By the induction hypothesis, either $\ttt{t}_1$ is
  a value or else it can make a step of evaluation, and likewise
  $\ttt{t}_2$. If $\ttt{t}_1$ can take a step, then rule \tsc{E-App1}
  applies to $\ttt{t}$. If $\ttt{t}_1$ is a value and $\ttt{t}_2$ can
  take a step, then rule \tsc{E-App2} applies. Finally, if both
  $\ttt{t}_1$ and $\ttt{t}_2$ are values, then $\ttt{t}_1$ must be an
  abstraction of the form
  $\ttt{$\lambda$x:$\ttt{T}_{11}$.$\ttt{t}_{12}$}$, and so rule
  \tsc{E-AppAbs} applies to $\ttt{t}$.
\end{proof}

Our next job is to prove that evaluation preserves types. We begin by
stating a couple of ``structural lemmas'' for the typing relation.
These are not particularly interesting in themselves, but will permit
us to perform some useful manipulations of typing derivations in later proofs.

\begin{lemma}[Permutation]
  If $\Gamma \vdash \ttt{t} : \ttt{T}$ and $\Delta$ is a permutation of
  $\Gamma$, then $\Delta \vdash \ttt{t} : \ttt{T}$. Moreover, the
  latter derivation has the same depth as the former.
\end{lemma}

\begin{proof}
  Straightforward induction on typing derivations.
\end{proof}

\begin{lemma}[Weakening]
  If $\Gamma \vdash \ttt{t} : \ttt{T}$ and $\ttt{x} \notin \text{dom}(\Gamma)$,
  then $\Gamma, \ttt{x:S} \vdash \ttt{t} : \ttt{T}$ for any type
  $\ttt{S}$. Moreover, the latter derivation has the same depth as the former.
\end{lemma}

\begin{proof}
  Straightforward induction on typing derivations.
\end{proof}

Using these technical lemmas, we can prove a crucial property of the
typing relation: that well-typedness is preserved when variables are
substituted with terms of appropriate types.

\begin{lemma}[Preservation of types under substitution]
  \lemlabel{substitution-lemma-for-simply-typed-lambda-calculus}
  If $\Gamma, \ttt{x:S} \vdash \ttt{t} : \ttt{T}$ and $\Gamma \vdash
  \ttt{s} : \ttt{S}$, then $\Gamma \vdash \ttt{[x$\mapsto$s]t} : \ttt{T}$.
\end{lemma}

\begin{proof}
  By induction on a derivation of the statement $\Gamma, \ttt{x:S}
  \vdash \ttt{t} : \ttt{T}$. For a given derivation, we proceed by
  cases on the final typing rule used in the proof. The most
  interesting cases are the ones for variables and abstractions.

  Case \tsc{T-Var}: $\ttt{t} = \ttt{z}$ with $\ttt{z:T} \in (\Gamma,
  \ttt{x:S})$. There are two subcases to consider, depending on
  whether $\ttt{z}$ is $\ttt{x}$ or another variable. If $\ttt{z} =
  \ttt{x}$, then $\ttt{[x$\mapsto$s]t} = \ttt{s}$. The required
  result is then $\Gamma \vdash \ttt{s} : \ttt{S}$, which is among
  the assumptions of the lemma. Otherwise, $\ttt{[x$\mapsto$s]t} =
  \ttt{z}$, and the desired result is immediate.

  Case \tsc{T-Abs}: $\ttt{t} =
  \ttt{$\lambda$y:$\ttt{T}_2$.$\ttt{t}_1$}$, $\ttt{T} =
  \ttt{$\ttt{T}_2$$\rightarrow$$\ttt{T}_1$}$, and $\Gamma,
  \ttt{x:S}, \ttt{y:$\ttt{T}_2$} \vdash \ttt{t}_1 : \ttt{T}_1$. We
  may assume $\ttt{x} \neq \ttt{y}$ and $\ttt{y} \notin FV(\ttt{s})$.
  Using permutation on the given subderivation, we obtain $\Gamma,
  \ttt{y:$\ttt{T}_2$}, \ttt{x:S} \vdash \ttt{t}_1 : \ttt{T}_1$. Using
  weakening on the other given derivation ($\Gamma \vdash s : S$), we
  obtain $\Gamma, \ttt{y:$\ttt{T}_2$} \vdash \ttt{s} : \ttt{S}$. Now,
  by the induction hypothesis, we have $\Gamma, \ttt{y:$\ttt{T}_2$}
  \vdash \ttt{[x$\mapsto$s]t}_1 : \ttt{T}_1$. Finally, we can apply
  the rule \tsc{T-Abs} to conclude that $\Gamma \vdash
  \ttt{$\lambda$y:$\ttt{T}_2$.$\ttt{[x$\mapsto$s]t}_1$} :
  \ttt{$\ttt{T}_2$$\rightarrow$$\ttt{T}_1$}$, which is the desired
  result since $\ttt{[x$\mapsto$s]t} =
  \ttt{$\lambda$y:$\ttt{T}_2$.$\ttt{[x$\mapsto$s]t}_1$}$ and $\ttt{T}
  = \ttt{$\ttt{T}_2$$\rightarrow$$\ttt{T}_1$}$.

  Case \tsc{T-App}: $\ttt{t} = \ttt{$\ttt{t}_1$\,$\ttt{t}_2$}$, $\Gamma,
  \ttt{x:S} \vdash \ttt{t}_1 :
  \ttt{$\ttt{T}_2$$\rightarrow$$\ttt{T}_1$}$, $\Gamma, \ttt{x:S}
  \vdash \ttt{t}_2 : \ttt{T}_2$, and $\ttt{T} = \ttt{T}_1$. By the
  induction hypothesis, we have $\Gamma, \ttt{x:S} \vdash
  \ttt{[x$\mapsto$s]t}_1 : \ttt{$\ttt{T}_2$$\rightarrow$$\ttt{T}_1$}$
  and $\Gamma, \ttt{x:S} \vdash \ttt{[x$\mapsto$s]t}_2 : \ttt{T}_2$.
  By the rule \tsc{T-App}, we can conclude that $\Gamma \vdash
  \ttt{([x$\mapsto$s]$\ttt{t}_1$)\,([x$\mapsto$s]$\ttt{t}_2$)} :
  \ttt{T}$. That is, $\Gamma \vdash
  \ttt{[x$\mapsto$s]($\ttt{t}_1$\,$\ttt{t}_2$)} : \ttt{T}$, which is
  the desired result.
\end{proof}

\begin{theorem}[Preservation]
  If $\Gamma \vdash \ttt{t} : \ttt{T}$ and $\ttt{t} \rightarrow
  \ttt{t}'$, then $\Gamma \vdash \ttt{t}' : \ttt{T}$.
\end{theorem}

\begin{proof}
  By induction on a derivation of $\Gamma \vdash \ttt{t} : \ttt{T}$.
  At each step of the induction, we assume that the desired property
  holds for all subderivations (i.e. that if $\Gamma \vdash \ttt{s} :
    \ttt{S}$ and $\ttt{s} \rightarrow \ttt{s}'$, then $\Gamma \vdash
    \ttt{s}' : \ttt{S}$, whenever $\Gamma \vdash \ttt{s} : \ttt{S}$ is
  proved by a subderivation of the present one) and proceed by case
  analysis on the last rule used in the derivation.

  Case \tsc{T-Var}: $\ttt{t} = \ttt{x}$ with $\ttt{x:T} \in \Gamma$.
  Can't happen.

  Case \tsc{T-Abs}: $\ttt{t} =
  \ttt{$\lambda$x:$\ttt{T}_1$.$\ttt{t}_2$}$. Can't happen.

  Case \tsc{T-App}: $\ttt{t} = \ttt{$\ttt{t}_1$\,$\ttt{t}_2$}$, $\Gamma
  \vdash \ttt{t}_1 :
  \ttt{$\ttt{T}_{11}$$\rightarrow$$\ttt{T}_{12}$}$, $\Gamma \vdash
  \ttt{t}_2 : \ttt{T}_{11}$, and $\ttt{T} = \ttt{T}_{12}$. There are
  three rules by which $\ttt{t} \rightarrow \ttt{t}'$ can be derived:
  \tsc{E-App1}, \tsc{E-App2}, and \tsc{E-AppAbs}. We consider each
  case separately.
  \begin{itemize}
    \item Subcase \tsc{E-App1}: $\ttt{t}_1 \rightarrow \ttt{t}_1'$
      and $\ttt{t}' = \ttt{$\ttt{t}_1'$\,$\ttt{t}_2$}$. From the assumptions of
      the \tsc{T-App} case, we have a subderivation of the original
      typing derivation whose conclusion is $\Gamma \vdash \ttt{t}_1
      : \ttt{$\ttt{T}_{11}$$\rightarrow$$\ttt{T}_{12}$}$. We can
      apply the induction hypothesis to this subderivation, obtaining
      $\Gamma \vdash \ttt{t}_1' :
      \ttt{$\ttt{T}_{11}$$\rightarrow$$\ttt{T}_{12}$}$. Combining
      this with the facts (also from the assumptions of the
      \tsc{T-App} case) that $\Gamma \vdash \ttt{t}_2 :
      \ttt{T}_{11}$, we can apply rule \tsc{T-App} to conclude that
      $\Gamma \vdash \ttt{t}' : \ttt{T}$.
    \item Subcase \tsc{E-App2}: $\ttt{t}_1 = \ttt{v}_1$, $\ttt{t}_2
      \rightarrow \ttt{t}_2'$, and $\ttt{t}' = \ttt{v}_1 \ttt{t}_2'$. Similar.
    \item Subcase \tsc{E-AppAbs}: $\ttt{t}_1 =
      \ttt{$\lambda$x:$\ttt{T}_{11}$.$\ttt{t}_{12}$}$, $\ttt{t}_2 =
      \ttt{v}_2$, and $\ttt{t}' =
      \ttt{[x$\mapsto$$\ttt{v}_2$]$\ttt{t}_{12}$}$. We can
      deconstruct the typing derivation for
      $\ttt{$\lambda$x:$\ttt{T}_{11}$.$\ttt{t}_{12}$}$, yielding
      $\Gamma, \ttt{x:$\ttt{T}_{11}$} \vdash \ttt{t}_{12} :
      \ttt{T}_{12}$. From this and the substitution lemma
      (\lemref{substitution-lemma-for-simply-typed-lambda-calculus}), we
      obtain $\Gamma \vdash \ttt{t}' : \ttt{T}_{12}$.
  \end{itemize}
\end{proof}

\chapter{Simple Extensions}

\section{Base Types}
Every programming language provides a variety of \textit{base types}
plus appropriate primitive operations to manipulate these values. For
theoretical purposes, it is often useful to abstract away from the
details of particular base types and their operations, and instead
simply suppose that our language comes equipped with some set
$\mathcal{A}$ of \textit{uninterpreted} base types, with no primitive
operations at all. This is accomplished simply by including the
elements of $\mathcal{A}$ in the set of types, as shown below.

New syntactic forms:
\begin{align*}
  \ttt{T} ::= & \quad \dots && \text{type} \\
  & \quad \ttt{A} && \text{base type}
\end{align*}

\section{The Unit Type}
Another useful base type, found especially in languages in the ML
family, is the singleton type $\ttt{Unit}$, described below.

New syntactic forms:
\begin{align*}
  \ttt{t} ::= & \quad \dots && \text{term} \\
  & \quad \ttt{unit} && \text{constant unit} \\
  \\
  \ttt{v} ::= & \quad \dots && \text{value} \\
  & \quad \ttt{unit} && \text{constant unit} \\
  \\
  \ttt{T} ::= & \quad \dots && \text{type} \\
  & \quad \ttt{Unit} && \text{unit type}
\end{align*}

New typing rules:
\begin{prooftree}
  \AxiomC{$\Gamma \vdash \ttt{unit} : \ttt{Unit}$ (\tsc{T-Unit})}
\end{prooftree}

In contrast to the uninterpreted base types, the unit type is
interpreted in the simplest possible way: we explicitly introduce a
single element, the term constant $\ttt{unit}$ and a typing rule
making $\ttt{unit}$ an element of $\ttt{Unit}$. We also add
$\ttt{unit}$ to the set of possible result values of computations.

\section{Derived Forms: Sequencing and Wildcards}
In languages with side effects, it is often useful to evaluate two or
more expressions in sequence. The sequencing notation
$\ttt{$\ttt{t}_1$;$\ttt{t}_2$}$ has the effect of evaluating
$\ttt{t}_1$, throwing away its trivial result, and going on to
evaluate $\ttt{t}_2$.

There are actually two different ways to formalize sequencing. One is
to follow the same pattern we haved used for syntactic forms: add
$\ttt{$\ttt{t}_1$;$\ttt{t}_2$}$ as a new alternative in the syntax of
terms, and then add two evaluation rules
\begin{prooftree}
  \AxiomC{$\ttt{t}_1 \rightarrow \ttt{t}_1'$}
  \RightLabel{(\tsc{E-Seq})}
  \UnaryInfC{$\ttt{$\ttt{t}_1$;$\ttt{t}_2$} \rightarrow
  \ttt{$\ttt{t}_1'$;$\ttt{t}_2$}$}
\end{prooftree}

\begin{prooftree}
  \AxiomC{$\ttt{unit;$\ttt{t}_2$} \rightarrow \ttt{t}_2$ (\tsc{E-SeqNext})}
\end{prooftree}

and a typing rule
\begin{prooftree}
  \AxiomC{$\Gamma \vdash \ttt{t}_1 : \ttt{Unit}$}
  \AxiomC{$\Gamma \vdash \ttt{t}_2 : \ttt{T}_2$}
  \RightLabel{(\tsc{T-Seq})}
  \BinaryInfC{$\Gamma \vdash \ttt{$\ttt{t}_1$;$\ttt{t}_2$} : \ttt{T}_2$}
\end{prooftree}

capturing the intended behavior of $\ttt{;}$.

An alternative way of formalizing sequencing is simply to regard
$\ttt{$\ttt{t}_1$;$\ttt{t}_2$}$ as an abbreviation for the term
$\ttt{($\lambda$x:Unit.$\ttt{t}_2$)$\ttt{t}_1$}$, where the variable
$\ttt{x}$ is chosen fresh, i.e. different from all the free variables
of $\ttt{t}_2$.

It is intuitively fairly clear that these two presentations of
sequencing add up to the same thing as far as the programmer is
concerned: the high-level typing and evaluation rules for sequencing
can be derived from the abbreviation of
$\ttt{$\ttt{t}_1$;$\ttt{t}_2$}$ as
$\ttt{($\lambda$x.Unit.$\ttt{t}_2$)$\ttt{t}_1$}$. This intuitive
correspondence is captured more formally by arguing that typing and
evaluation both ``commute'' with the expansion of the abbreviation.

New derived forms:
\begin{align*}
  \ttt{$\ttt{t}_1$;$\ttt{t}_2$} \overset{\text{def}}{=}
  \ttt{($\lambda$x:Unit.$\ttt{t}_2$)\,$\ttt{t}_1$} \text{ where
  $\ttt{x} \notin FV(\ttt{t}_2)$}
\end{align*}

\section{Let Bindings}
When writing a complex expression, it is often useful to give names
to some of its subexpressions. Most languages provide one or more
ways of doing this. In ML, for example, we write $\ttt{let
x=$\ttt{t}_1$ in $\ttt{t}_2$}$ to mean ``evaluate expression
$\ttt{t}_1$ and bind the name $\ttt{x}$ to the resulting value, while
evaluating $\ttt{t}_2$.''

New syntactic forms:
\begin{align*}
  \ttt{t} ::= & \quad \dots && \text{term} \\
  & \quad \ttt{let x=t in t} && \text{let binding}
\end{align*}

New evaluation rules:
\begin{prooftree}
  \AxiomC{$\ttt{let x=$\ttt{v}_1$ in $\ttt{t}_2$} \rightarrow
  \ttt{[x$\mapsto$$\ttt{v}_1$]}\ttt{t}_2$ (\tsc{E-LetV})}
\end{prooftree}

\begin{prooftree}
  \AxiomC{$\ttt{t}_1 \rightarrow \ttt{t}_1'$}
  \RightLabel{(\tsc{E-Let})}
  \UnaryInfC{$\ttt{let x=$\ttt{t}_1$ in $\ttt{t}_2$} \rightarrow
  \ttt{let x=$\ttt{t}_1'$ in $\ttt{t}_2$}$}
\end{prooftree}

New typing rules:
\begin{prooftree}
  \AxiomC{$\Gamma \vdash \ttt{t}_1 : \ttt{T}_1$}
  \AxiomC{$\Gamma, \ttt{x:$\ttt{T}_1$} \vdash \ttt{t}_2 : \ttt{T}_2$}
  \RightLabel{(\tsc{T-Let})}
  \BinaryInfC{$\Gamma \vdash \ttt{let x=$\ttt{t}_1$ in $\ttt{t}_2$}
  : \ttt{T}_2$}
\end{prooftree}

Our $\ttt{let}$-binder follows ML's in choosing a call-by-value
evaluation order, where the $\ttt{let}$-bound term must be fully
evaluated before evaluation of the $\ttt{let}$-body can begin. The
typing rule $\tsc{T-Let}$ tells us that the type of a $\ttt{let}$ can
be calculated the type of the $\ttt{let}$-bound term, extending the
context with a binding with this type, and in this enriched context
calculating the type of the body, which is then the type of the whole
$\ttt{let}$ expression.

Can $\ttt{let}$ also be defined as a derived form? Yes, as Landin
showed, but the details are slightly more subtle than what we did for
sequencing an ascripiton. Naively, it is clear that we use a
combination of abstraction and application to achieve the effect of a
$\ttt{let}$-binding:
\[
  \ttt{let x=$\ttt{t}_1$ in $\ttt{t}_2$} \overset{\text{def}}{=}
  \ttt{($\lambda$x:$\ttt{T}_1$.$\ttt{t}_2$)\,$\ttt{t}_1$}
\]

But notice that the right-hand side of this abbreviation includes the
type annotation $\ttt{T}_1$, which does not appear on the left-hand
side. That is, if we imagine a derived forms as being desugared
during the parsing phase of some compiler, then we need to ask how
the parser is supposed to know that it should generate $\ttt{T}_1$ as
the type annotation on the $\lambda$ in the desugared internal language term.

The answer, of course, is that this information comes from the
typechecker! We discover the needed type annotation simply by
calculating the type of $\ttt{t}_1$. More formally, what this tells
us is that the $\ttt{let}$ constructor is a slightly different sort
of derived form than the ones we have seen so far: we should regard
it as a transformation on the \textit{typing derivations} that maps a
derivation involving $\ttt{let}$:
\begin{prooftree}
  \AxiomC{$\vdots$}
  \UnaryInfC{$\Gamma \vdash \ttt{t}_1 : \ttt{T}_1$}
  \AxiomC{$\vdots$}
  \UnaryInfC{$\Gamma, \ttt{x:$\ttt{T}_1$} \vdash \ttt{t}_2 : \ttt{T}_2$}
  \RightLabel{(\tsc{T-Let})}
  \BinaryInfC{$\Gamma \vdash \ttt{let x=$\ttt{t}_1$ in $\ttt{t}_2$}
  : \ttt{T}_2$}
\end{prooftree}
to one using abstraction and application:
\begin{prooftree}
  \AxiomC{$\vdots$}
  \UnaryInfC{$\Gamma, \ttt{x:$\ttt{T}_1$} \vdash \ttt{t}_2 : \ttt{T}_2$}
  \RightLabel{(\tsc{T-Abs})}
  \UnaryInfC{$\Gamma \vdash \ttt{$\lambda$x:$\ttt{T}_1$.$\ttt{t}_2$}
  : \ttt{$\ttt{T}_1$$\rightarrow$$\ttt{T}_2$}$}
  \AxiomC{$\vdots$}
  \UnaryInfC{$\Gamma \vdash \ttt{t}_1 : \ttt{T}_1$}
  \RightLabel{(\tsc{T-App})}
  \BinaryInfC{$\Gamma \vdash
  \ttt{($\lambda$x:$\ttt{T}_1$.$\ttt{t}_2$)\,$\ttt{t}_1$} : \ttt{T}_2$}
\end{prooftree}
Thus, $\ttt{let}$ is a little less derived than the other derived
forms we have seen: we can derive its evalution behavior by
desugaring it, but its typing behavior must be built into the internal language.

\section{General Recursion}
Another facility found in most programming languages is the ability
to define recursive functions. Such functions can be defined with the
aid of the \vocab{fixed-point combinator}:
\[ \ttt{fix =
  $\lambda$f.($\lambda$x.f($\lambda$y.x\,x\,y))\,($\lambda$x.f($\lambda$y.x\,x\,y))}
\]

Suppose we want to write a recursive function definition of the form
$\ttt{h = $\langle$\textnormal{body containing }h$\rangle$}$. The
intention is that the recursive definition should be ``unrolled'' at
the point where it occurs; for example, the definition of
$\ttt{factorial}$ would intuitively be:
\begin{verbatim}
  if n = 0 then 1
  else n * (if (n-1)=0 then 1
            else (n-1) * (if (n-2)=0 then 1
                          else (n-2) * ...))
\end{verbatim}
This effect can be achieved using the $\ttt{fix}$ combinator by first
defining $\ttt{g = $\lambda$f.$\langle$\textnormal{body containing
}f$\rangle$}$ and then $\ttt{h = fix g}$. For example, we can define
the factorial function by
\begin{align*}
  & \ttt{g = $\lambda$fct.($\lambda$n.(if n = 0 then 1 else n * (fct
  (n-1))))} \\
  & \ttt{factorial = fix g}
\end{align*}

Recursive functions can be defined in a typed setting in a similar
way. However, there is one important difference: $\ttt{fix}$ itself
cannot be defined in the simply typed lambda calculus. Indeed, we
will see in \chpref{normalization} that expression that can lead to
non-terminating computations cannot be typed using only simple types.
So, instead of defining $\ttt{fix}$ as a term in the language, we
simply add it as a new primitive, with evaluatiion rules mimicking
the behavior of the untyped $\ttt{fix}$ combinator and a typing rule
that captures its intended uses.

New syntactic forms:
\begin{align*}
  \ttt{t} ::= & \quad \dots && \text{term} \\
  & \quad \ttt{fix t} && \text{fixed point of \ttt{t}}
\end{align*}

New evaluation rules:
\begin{prooftree}
  \AxiomC{$\ttt{fix ($\lambda$x:$\ttt{T}_1$.$\ttt{t}_2$)} \rightarrow
    \ttt{[x$\mapsto$(fix
  ($\lambda$x:$\ttt{T}_1$.$\ttt{t}_2$))]$\ttt{t}_2$}$ (\tsc{E-FixBeta})}
\end{prooftree}

\begin{prooftree}
  \AxiomC{$\ttt{t}_1 \rightarrow \ttt{t}_1'$}
  \RightLabel{(\tsc{E-Fix})}
  \UnaryInfC{$\ttt{fix $\ttt{t}_1$} \rightarrow \ttt{fix $\ttt{t}_1'$}$}
\end{prooftree}

New typing rules:
\begin{prooftree}
  \AxiomC{$\Gamma \vdash \ttt{t}_1 : \ttt{T}_1 \rightarrow \ttt{T}_1$}
  \RightLabel{(\tsc{T-Fix})}
  \UnaryInfC{$\Gamma \vdash \ttt{fix $\ttt{t}_1$} : \ttt{T}_1$}
\end{prooftree}

New derived forms:
\begin{align*}
  \ttt{letrec x:$\ttt{T}_1$=$\ttt{t}_1$ in $\ttt{t}_2$} \overset{\text{def}}{=}
  \ttt{let x=fix($\lambda$x:$\ttt{T}_1$.$\ttt{t}_1$) in $\ttt{t}_2$}
\end{align*}

The simply typed lambda calculus with numbers and $\ttt{fix}$ has
long been the favorite experimental subject of programming language
researchers, since it is the simplest language in which a range of
subtle semantics phenomena such as \textit{full abstraction} arise.
It is often called PCF.

\chapter{Normalization}
\chplabel{normalization}
In this chapter, we consider another fundamental property of the pure
simply typed lambda calculus: the fact that the evaluation of a
well-typed program is guaranteed to halt in a finite number of steps.
That is, every well-typed term is \textit{normalizable}.

\section{Normalization for Simple Types}
The calculus we shall consider here is the simply typed lambda
calculus over a single base type $\ttt{A}$. Normalization for this
calculus is not entirely trivial to prove, since each reduction of a
term can duplicate redexes in subterms.

\begin{exercise}
  Where do we fail if we attempt to prove normalization by a
  straightforward induction on the size of a well-typed term?

  We fail because of the substitution that occurs during
  $\beta$-reduction (\tsc{E-AppAbs}). When an application of the
  form $\ttt{($\lambda$x:$\ttt{T}_{11}$.$\ttt{t}_{12}$)\,$\ttt{v}_2$}$
  reduces to $\ttt{[x$\mapsto$$\ttt{v}_2$]}\ttt{t}_{12}$, the
  resulting term may be larger than the original term. Consider this example:
  \begin{itemize}
    \item Original term:
      $\ttt{($\lambda$x:A.x\,x\,x)($\lambda$y:A.y\,y)}$ (11 nodes)
    \item After reduction:
      $\ttt{($\lambda$y:A.y\,y)($\lambda$y:A.y\,y)($\lambda$y:A.y\,y)}$
      (14 nodes)
  \end{itemize}
\end{exercise}

The key issue here (as in many proofs by induction) is finding a
strong enough induction hypothesis.

\begin{definition}
  \deflabel{normalization-relation}
  We define, for each type $\ttt{T}$, a set $\mathcal{R}_\ttt{T}$ of
  closed terms of type $\ttt{T}$ as follows:
  \begin{itemize}
    \item For the base type $\ttt{A}$, $\mathcal{R}_\ttt{A}(\ttt{t})$
      holds if and only if $\ttt{t}$ halts.
    \item For function types, $\mathcal{R}_{\ttt{T}_1 \rightarrow
      \ttt{T}_2}(\ttt{t})$ holds if and only if $\ttt{t}$ halts and
      for every $\ttt{s}$ such that
      $\mathcal{R}_{\ttt{T}_1}(\ttt{s})$ holds, we also have
      $\mathcal{R}_{\ttt{T}_2}(\ttt{t\,s})$.
  \end{itemize}
\end{definition}

This gives us the strengthened induction hypothesis that we need. Our
primary goal is to show that all programs (i.e. all closed terms of
base type) halt. But closed terms of base type can contain subterms
of functional type, so we need to know something about these as well.
Moreover, it is not enough to know that these subterms halt, because
the application of a normalized function to a normalized argument
involves a substitution, which may enable more evaluation steps. So
we need a stronger condition for terms of functional type: not only
should they halt themselves, but, when applied to halting arguments,
they should yield halting results.

The form of \defref{normalization-relation} is characteristic of the
\textit{logical relations} proof technique. (Since we are just
  dealing with unary relations here, we should more properly say
\textit{logical predicates}.) If we want to prove some property
$\mathcal{P}$ of all closed terms of type $\ttt{A}$, we proceed by
proving, by induction on types, that all terms of type $\ttt{A}$
\textit{possess} property $\mathcal{P}$, all terms of type
$\ttt{A$\rightarrow$A}$ \textit{preserve} property $\mathcal{P}$, all
terms of type $\ttt{(A$\rightarrow$A)$\rightarrow$(A$\rightarrow$A)}$
\textit{preserve the property of preserving property} $\mathcal{P}$,
and so on. We do this by defining a family of predicates, indexed by
types. For the base type $\ttt{A}$, the predicate is just
$\mathcal{P}$. For functional types, it says that the function should
map values satisfying the predicate at the input type to values
satisfying the predicate at the output type.

We use this definition to carry out the proof the proof of
normalization in two steps. First, we observe that every element of
the set $\mathcal{R}_\ttt{T}$ is normalizable. Then we show that
every well-typed term of type $\ttt{T}$ is an element of $\mathcal{R}_\ttt{T}$.

The first step is immediate from the definition of $\mathcal{R}_\ttt{T}$.

\begin{lemma}
  \lemlabel{normalization-relation-implies-normalization}
  If $\mathcal{R}_\ttt{T}(\ttt{t})$, then $\ttt{t}$ halts.
\end{lemma}

The second step is broken down into two lemmas. First, we remark that
membership in $\mathcal{R}_\ttt{T}$ is invariant under evaluation.

\begin{lemma}
  \lemlabel{normalization-relation-invariance}
  If $\ttt{t} : \ttt{T}$ and $\ttt{t} \rightarrow \ttt{t}'$, then
  $\mathcal{R}_\ttt{T}(\ttt{t})$ if and only if
  $\mathcal{R}_\ttt{T}(\ttt{t}')$.
\end{lemma}

\begin{proof}
  By induction on the structure of the type $\ttt{T}$. Note, first,
  that it is clear that $\ttt{t}$ halts if and only if $\ttt{t}'$ does.
  \begin{itemize}
    \item If $\ttt{T} = \ttt{A}$, then there is nothing more to show.
    \item If $\ttt{T} = \ttt{$\ttt{T}_1$$\rightarrow$$\ttt{T}_2$}$ for
      some $\ttt{T}_1$ and $\ttt{T}_2$. There are two directions:

      ($\Rightarrow$) $\mathcal{R}_\ttt{T}(\ttt{t})$, i.e. $\ttt{t}$ halts
      and $\mathcal{R}_{\ttt{T}_2}(\ttt{t\,s})$ whenever
      $\mathcal{R}_{\ttt{T}_1}(\ttt{s})$ for some arbitrary $\vdash
      \ttt{s} : \ttt{T}_1$. Inductive hypothesis: If $\vdash \ttt{t}
      : \ttt{T}_2$ and $\ttt{t} \rightarrow \ttt{t}'$, then
      $\mathcal{R}_{\ttt{T}_2}(\ttt{t})$ iff
      $\mathcal{R}_{\ttt{T}_2}(\ttt{t}')$. Our premise tells us that
      $\ttt{t} \rightarrow \ttt{t}'$, so we have $\ttt{t\,s}
      \rightarrow \ttt{$\ttt{t}'$\,s}$. Applying the inductive
      hypothesis gives us
      $\mathcal{R}_{\ttt{T}_2}(\ttt{$\ttt{t}'$\,s})$. Since this
      holds for arbitrary $\ttt{s}$, the definition of
      $\mathcal{R}_\ttt{T}$ gives us $\mathcal{R}_\ttt{T}(\ttt{t}')$.

      ($\Leftarrow$) Analogous.
  \end{itemize}
\end{proof}

Next, we want to show that every term of type $\ttt{T}$ belongs to
$\mathcal{R}_\ttt{T}$. Here, the induction will be on typing
derivations (it would be surprising to see a proof about well-typed
  terms that did not somewhere involve induction on typing
derivations!). The only technical difficulty here is in dealing with
the $\lambda$-abstraction case. Since we are arguing by induction,
the demonstration that a term
$\ttt{$\lambda$x:$\ttt{T}_1$.$\ttt{t}_2$}$ belongs to
$\mathcal{R}_\ttt{$\ttt{T}_1$$\rightarrow$$\ttt{T}_2$}$ should
involve applying the induction hypothesis to show that $\ttt{t}_2$
belongs to $\mathcal{R}_{\ttt{T}_2}$. But $\mathcal{R}_{\ttt{T}_2}$
is defined to be a set of \textit{closed} terms, while $\ttt{t}_2$
may contain free $\ttt{x}$, so this does not make sense.

This problem is resolved by using a standard trick to suitably
generalize the induction hypothesis: instead of proving a statement
involving a closed term, we generalize it to cover all closed
\textit{instances} of an open term $\ttt{t}$.

\begin{lemma}
  \lemlabel{normalization-for-open-terms}
  If $\ttt{$\ttt{x}_1$:$\ttt{T}_1$}, \dots, \ttt{$\ttt{x}_n$:$\ttt{T}_n$} \vdash
  \ttt{t} : \ttt{T}$ and $\ttt{v}_1, \dots, \ttt{v}_n$ are closed
  values of types $\ttt{T}_1, \dots, \ttt{T}_n$ with
  $\mathcal{R}_{\ttt{T}_i}(\ttt{v}_i)$ for each $i$, then
  $\mathcal{R}_\ttt{T}(\ttt{[$\ttt{x}_1$$\mapsto$$\ttt{v}_1$]$\cdots$[$\ttt{x}_n$$\mapsto$$\ttt{v}_n$]t})$.
\end{lemma}

\begin{proof}
  By induction on a derivation of $\ttt{$\ttt{x}_1$:$\ttt{T}_1$},
  \dots, \ttt{$\ttt{x}_n$:$\ttt{T}_n$} \vdash \ttt{t} : \ttt{T}$.
  (The most interesting case is the one for abstraction.)

  Case \tsc{T-Var}: $\ttt{t} = \ttt{x}_i$ and $\ttt{T} = \ttt{T}_i$
  for some $i$. Immediate.

  Case \tsc{T-Abs}: $\ttt{t} =
  \ttt{$\lambda$x:$\ttt{S}_1$.$\ttt{s}_2$}$, \quad
  $\ttt{$\ttt{x}_1$:$\ttt{T}_1$}, \dots,
  \ttt{$\ttt{x}_n$:$\ttt{T}_n$}, \ttt{$\ttt{x}$:$\ttt{S}_1$} \vdash
  \ttt{s}_2 : \ttt{S}_2$, \quad $\ttt{T} =
  \ttt{$\ttt{T}_1$$\rightarrow$$\ttt{T}_2$}$. Obviously,
  $\ttt{[$\ttt{x}_1$$\mapsto$$\ttt{v}_1$]$\cdots$[$\ttt{x}_n$$\mapsto$$\ttt{v}_n$]}t$
  evaluates to a value, since it is a value already. What remains to
  show is that
  $\mathcal{R}_{\ttt{S}_2}(\ttt{([$\ttt{x}_1$$\mapsto$$\ttt{v}_1$]$\cdots$[$\ttt{x}_n$$\mapsto$$\ttt{v}_n$]t)\,s})$
  for any $\ttt{s} : \ttt{S}_1$ such that
  $\mathcal{R}_{\ttt{S}_1}(\ttt{s})$. Suppose $\ttt{s}$ is such a
  term. By \lemref{normalization-relation-implies-normalization}, we
  have $\ttt{s} \rightarrow^\ast \ttt{v}$ for some $\ttt{v}$. By
  \lemref{normalization-relation-invariance},
  $\mathcal{R}_{\ttt{S}_1}(v)$. Now, by the inductive hypothesis,
  $\mathcal{R}_{\ttt{S}_2}(\ttt{[$\ttt{x}_1$$\mapsto$$\ttt{v}_1$]$\cdots$[$\ttt{x}_n$$\mapsto$$\ttt{v}_n$][x$\mapsto$v]$\ttt{s}_2$})$.
  Applying the evaluation rule \tsc{E-AppAbs} gives us
  \[
    \ttt{($\lambda$x:$\ttt{S}_1$.[$\ttt{x}_1$$\mapsto$$\ttt{v}_1$]$\cdots$[$\ttt{x}_n$$\mapsto$$\ttt{v}_n$][x$\mapsto$v]$\ttt{s}_2$)\,s}
    \rightarrow^\ast
    \ttt{[$\ttt{x}_1$$\mapsto$$\ttt{v}_1$]$\cdots$[$\ttt{x}_n$$\mapsto$$\ttt{v}_n$][x$\mapsto$v]$\ttt{s}_2$}
  \]
  from which \lemref{normalization-relation-invariance} gives us
  \[
    \mathcal{R}_{\ttt{S}_2}(\ttt{($\lambda$x:$\ttt{S}_1$.[$\ttt{x}_1$$\mapsto$$\ttt{v}_1$]$\cdots$[$\ttt{x}_n$$\mapsto$$\ttt{v}_n$][x$\mapsto$v]$\ttt{s}_2$)\,s}).
  \]
  That is,
  \[
    \mathcal{R}_{\ttt{S}_2}(\ttt{([$\ttt{x}_1$$\mapsto$$\ttt{v}_1$]$\cdots$[$\ttt{x}_n$$\mapsto$$\ttt{v}_n$][x$\mapsto$v]($\lambda$x:$\ttt{S}_1$.$\ttt{s}_2$))\,s}).
  \]
  Since $\ttt{s}$ was chosen arbitrarily, the definition of
  $\mathcal{R}_{\ttt{$\ttt{S}_1$$\rightarrow$$\mathcal{S}_2$}}$ gives us
  \[
    \mathcal{R}_{\ttt{$\ttt{S}_1$$\rightarrow$$\ttt{S}_2$}}(\ttt{[$\ttt{x}_1$$\mapsto$$\ttt{v}_1$]$\cdots$[$\ttt{x}_n$$\mapsto$$\ttt{v}_n$]($\lambda$x:$\ttt{S}_1$.$\ttt{s}_2$)}).
  \]

  Case \tsc{T-App}: $\ttt{t} = \ttt{t}_1\,\ttt{t}_2$, \quad
  $\ttt{$\ttt{x}_1$:$\ttt{T}_1$}, \dots,
  \ttt{$\ttt{x}_n$:$\ttt{T}_n$} \vdash \ttt{t}_1 :
  \ttt{$\ttt{T}_{11}$$\rightarrow$$\ttt{T}_{12}$}$, \quad
  $\ttt{$\ttt{x}_1$:$\ttt{T}_1$}, \dots,
  \ttt{$\ttt{x}_n$:$\ttt{T}_n$} \vdash \ttt{t}_2 : \ttt{T}_{11}$,
  \quad $\ttt{T} = \ttt{T}_{12}$. The inductive hypothesis gives us
  $\mathcal{R}_{\ttt{$\ttt{T}_{11}$$\rightarrow$$\ttt{T}_{12}$}}(\ttt{[$\ttt{x}_1$$\mapsto$$\ttt{v}_1$]$\cdots$[$\ttt{x}_n$$\mapsto$$\ttt{v}_n$]$\ttt{t}_1$})$
  and
  $\mathcal{R}_{\ttt{T}_{11}}(\ttt{[$\ttt{x}_1$$\mapsto$$\ttt{v}_1$]$\cdots$[$\ttt{x}_n$$\mapsto$$\ttt{v}_n$]$\ttt{t}_2$})$.
  By the definition of
  $\mathcal{R}_{\ttt{$\ttt{T}_{11}$$\rightarrow$$\ttt{T}_{12}$}}$, we have
  \[
    \mathcal{R}_{\ttt{T}_{12}}(\ttt{([$\ttt{x}_1$$\mapsto$$\ttt{v}_1$]$\cdots$[$\ttt{x}_n$$\mapsto$$\ttt{v}_n$]$\ttt{t}_1$)([$\ttt{x}_1$$\mapsto$$\ttt{v}_1$]$\cdots$[$\ttt{x}_n$$\mapsto$$\ttt{v}_n$]$\ttt{t}_2$)}).
  \]
  That is,
  \[
    \mathcal{R}_{\ttt{T}_{12}}(\ttt{[$\ttt{x}_1$$\mapsto$$\ttt{v}_1$]$\cdots$[$\ttt{x}_n$$\mapsto$$\ttt{v}_n$]($\ttt{t}_1$\,$\ttt{t}_2$)}).
  \]
\end{proof}

\begin{theorem}[Normalization]
  If $\vdash \ttt{t} : \ttt{T}$, then $\ttt{t}$ is normalizable.
\end{theorem}

\begin{proof}
  We obtain the normalization property as a corollary, simply by
  taking the term $\ttt{t}$ to be closed in
  \lemref{normalization-for-open-terms} and then recalling that all
  elements of $\mathcal{R}_\ttt{T}$ are normalizable by
  \lemref{normalization-relation-implies-normalization}.
\end{proof}

\end{document}
