%%fakesection Load packages

\usepackage{lmodern}
\usepackage[pdfusetitle]{hyperref}
\ExplSyntaxOn
\sys_if_engine_luatex:T {
  \usepackage{luatex85}
}
\sys_if_engine_pdftex:T {
  \usepackage[T1]{fontenc}
}
\ExplSyntaxOff

% These are evan.sty
\usepackage{amsmath,amssymb,amsthm}
\usepackage{mathrsfs}
\usepackage[usenames,svgnames,dvipsnames]{xcolor}
\usepackage{textcomp}
\usepackage{enumerate}
\usepackage[textsize=scriptsize,shadow]{todonotes}
\usepackage{mathtools}
\usepackage{microtype}
\usepackage[normalem]{ulem}
\usepackage{stmaryrd}
\usepackage{wasysym}
\usepackage{multirow}
\usepackage{prerex}
\usepackage[nameinlink]{cleveref}

%%fakesection evan.sty macros
%Small commands
%% Napkin commands
\newcommand{\prototype}[1]{
  \emph{{\color{red} Prototypical example for this section:} #1} \par\medskip
}
\newenvironment{moral}{%
  \begin{mdframed}[linecolor=green!70!black]%
  \bfseries\color{green!50!black}}%
  {
\end{mdframed}}

%%fakesection Links (hyperref loaded earlier implicitly)
\hypersetup{
  linkcolor={red!50!black},
  citecolor={green!50!black},
  urlcolor={blue!80!black},
  pdfkeywords={napkin,math},
  pdfsubject={web.evanchen.cc},
  colorlinks,
}

%%fakesection Commutative diagrams
\usepackage{tikz-cd}
\usetikzlibrary{arrows,arrows.meta}
% make a larger hook
% https://tex.stackexchange.com/questions/514451/how-to-define-a-new-hooked-arrow
\makeatletter
\pgfdeclarearrow{
  name=xGlyph,
  cache=false,
  bending mode=none,
  parameters={\tikzcd@glyph@len,\tikzcd@glyph@shorten},
  setup code={%
  \pgfarrowssettipend{\tikzcd@glyph@len\advance\pgf@x by\tikzcd@glyph@shorten}},
  defaults={
    glyph axis=axis_height,
    glyph length=+1.55ex,
  glyph shorten=+-0.1ex},
  drawing code={%
    \pgfpathrectangle{\pgfpoint{+0pt}{+-1.5ex}}{\pgfpoint{+\tikzcd@glyph@len}{+3ex}}%
    \pgfusepathqclip%
    \pgftransformxshift{+\tikzcd@glyph@len}%
    \pgftransformyshift{+-\tikzcd@glyph@axis}%
\pgftext[right,base]{\tikzcd@glyph}}}
\makeatother
\tikzcdset{
  arrow style=tikz,
  diagrams={>={Latex}},
  tikzcd left hook/.tip={xGlyph[glyph math command=supset, swap,
  glyph axis = 5.7pt]},
  tikzcd right hook/.tip={xGlyph[glyph math command=supset, glyph
  axis = 5.7pt]},
  surjective head arrow /.tip = {tikzcd to[sep=-1.5pt]tikzcd to},
  surjective head/.style={
    -surjective head arrow
  }
}

%%fakesection Page layout
\usepackage[headsepline]{scrlayer-scrpage}
\renewcommand{\headfont}{}
\addtolength{\textheight}{3.14cm}
\setlength{\footskip}{0.5in}
\setlength{\headsep}{10pt}

\def\shortdate{\leavevmode\hbox{\the\year-\twodigits\month-\twodigits\day}}
\def\twodigits#1{\ifnum#1<10 0\fi\the#1}
\automark[chapter]{chapter}

\rohead{\footnotesize\thepage}
\rehead{\footnotesize \textbf{\sffamily Napkin}, by \emph{Evan Chen}
(\napkinversion)}
\lehead{\footnotesize\thepage}
\lohead{\footnotesize \leftmark}
\chead{}
\rofoot{}
\refoot{}
\lefoot{}
\lofoot{}
%\cfoot{\pagemark}

%%fakesection Fancy section and chapter heads
\renewcommand*{\sectionformat}{\color{purple}\S\thesection\autodot\enskip}
\newcommand{\problemhead}{A few harder problems to think about}

\addtokomafont{chapterprefix}{\raggedleft}
\RedeclareSectionCommand[beforeskip=0.5em]{chapter}
\renewcommand*{\chapterformat}{%
  \mbox{\scalebox{1.5}{\chapappifchapterprefix{\nobreakspace}}%
\scalebox{2.718}{\color{purple}\thechapter\autodot}\enskip}}

\addtokomafont{partprefix}{\rmfamily}
\renewcommand*{\partformat}{\color{purple}\scalebox{2.5}{\thepart}}

%%fakesection Theorems
\usepackage{thmtools}
\usepackage[framemethod=TikZ]{mdframed}

\theoremstyle{definition}
\mdfdefinestyle{mdbluebox}{%
  linewidth=1pt,
  skipabove=12pt,
  innerbottommargin=9pt,
  skipbelow=2pt,
  nobreak=true,
  linecolor=blue,
  backgroundcolor=TealBlue!5,
}
\declaretheoremstyle[
  headfont=\sffamily\bfseries\color{MidnightBlue},
  mdframed={style=mdbluebox},
  headpunct={\\[3pt]},
  postheadspace={0pt}
]{thmbluebox}

\mdfdefinestyle{mdredbox}{%
  linewidth=0.5pt,
  skipabove=12pt,
  frametitleaboveskip=5pt,
  frametitlebelowskip=0pt,
  skipbelow=2pt,
  frametitlefont=\bfseries,
  innertopmargin=4pt,
  innerbottommargin=8pt,
  linecolor=RawSienna,
  backgroundcolor=Salmon!5,
}
\declaretheoremstyle[
  headfont=\bfseries\color{RawSienna},
  mdframed={style=mdredbox},
  headpunct={\\[3pt]},
  postheadspace={0pt},
]{thmredbox}

\declaretheorem[%
style=thmbluebox,name=Theorem,numberwithin=chapter]{theorem}
\declaretheorem[style=thmbluebox,name=Lemma,sibling=theorem]{lemma}
\declaretheorem[style=thmbluebox,name=Proposition,sibling=theorem]{proposition}
\declaretheorem[style=thmbluebox,name=Corollary,sibling=theorem]{corollary}
\declaretheorem[style=thmredbox,name=Example,sibling=theorem]{example}

\mdfdefinestyle{mdgreenbox}{%
  skipabove=8pt,
  linewidth=2pt,
  rightline=false,
  leftline=true,
  topline=false,
  bottomline=false,
  linecolor=ForestGreen,
  backgroundcolor=ForestGreen!5,
}
\declaretheoremstyle[
  headfont=\bfseries\sffamily\color{ForestGreen!70!black},
  bodyfont=\normalfont,
  spaceabove=2pt,
  spacebelow=1pt,
  mdframed={style=mdgreenbox},
  headpunct={ --- },
]{thmgreenbox}

\mdfdefinestyle{mdblackbox}{%
  skipabove=8pt,
  linewidth=2pt,
  rightline=false,
  leftline=true,
  topline=false,
  bottomline=false,
  linecolor=black,
  backgroundcolor=RedViolet!5!gray!5,
}
\declaretheoremstyle[
  headfont=\bfseries,
  bodyfont=\normalfont,
  spaceabove=2pt,
  spacebelow=1pt,
  mdframed={style=mdblackbox}
]{thmblackbox}

\declaretheorem[name=Question,sibling=theorem,style=thmblackbox]{ques}
\declaretheorem[name=Exercise,sibling=theorem,style=thmblackbox]{exercise}
\declaretheorem[name=Outline,sibling=theorem,style=thmblackbox]{outline}
\declaretheorem[name=Remark,sibling=theorem,style=thmgreenbox]{remark}
\declaretheorem[name=Step,style=thmgreenbox]{step} % only used in Lebesgue int

\definecolor{darkmagenta}{rgb}{0.55, 0.0, 0.55}
\definecolor{patriarch}{rgb}{0.5, 0.0, 0.5}

\theoremstyle{definition}
\mdfdefinestyle{mdpurplebox}{%
  linewidth=1pt,
  skipabove=12pt,
  innerbottommargin=9pt,
  skipbelow=2pt,
  nobreak=true,
  linecolor=darkmagenta,
  backgroundcolor=patriarch!5,
}
\declaretheoremstyle[
  headfont=\sffamily\bfseries\color{darkmagenta},
  mdframed={style=mdpurplebox},
  headpunct={\\[3pt]},
  postheadspace={0pt}
]{thmpurplebox}
\declaretheorem[style=thmpurplebox,name=Definition,sibling=theorem]{definition}
\newtheorem{claim}[theorem]{Claim}
\declaretheorem[style=thmbluebox,name=Fact,sibling=theorem]{fact}
\newtheorem{abuse}[theorem]{Abuse of Notation}

\definecolor{alizarin}{rgb}{0.82, 0.1, 0.26}

\theoremstyle{definition}
\mdfdefinestyle{mdhypbox}{%
  linewidth=1pt,
  skipabove=12pt,
  innerbottommargin=9pt,
  skipbelow=2pt,
  nobreak=true,
  linecolor=alizarin,
  backgroundcolor=Red!5,
}
\declaretheoremstyle[
  headfont=\sffamily\bfseries\color{alizarin},
  mdframed={style=mdhypbox},
  headpunct={\\[3pt]},
  postheadspace={0pt}
]{thmhypbox}
\declaretheorem[style=thmhypbox,name=Hypothesis,sibling=theorem]{hypothesis}

\newtheorem{problem}{Problem}[chapter]
\renewcommand{\theproblem}{\thechapter\Alph{problem}}
\newtheorem{sproblem}[problem]{Problem}
\newtheorem{dproblem}[problem]{Problem}
\renewcommand{\thesproblem}{\theproblem$^{\star}$}
\renewcommand{\thedproblem}{\theproblem$^{\dagger}$}
\newcommand{\listhack}{$\empty$\vspace{-2em}}

%%fakesection Answers
\usepackage{answers}
\Newassociation{hint}{answeritem}{tex/backmatter/all-hints}
\Newassociation{sol}{answeritem}{tex/backmatter/all-solns}
\renewcommand{\solutionextension}{out}
\renewenvironment{answeritem}[1]{
\item[\bfseries #1.]}{}

%%fakesection Table of contents
% First add ToC to ToC
\makeatletter
\usepackage{etoolbox}
\pretocmd{\tableofcontents}{%
  \if@openright\cleardoublepage\else\clearpage\fi
  \pdfbookmark[0]{\contentsname}{toc}%
}{}{}%
\makeatother
\setcounter{tocdepth}{1}
\RedeclareSectionCommand[tocnumwidth=4.2em]{part}
\RedeclareSectionCommand[tocpagenumberwidth=2.2em,tocnumwidth=4.2em]{chapter}
\RedeclareSectionCommand[tocpagenumberwidth=2.2em,tocnumwidth=2.8em]{section}
% adjust tocpagenumberwidth manually for large page number:
% https://tex.stackexchange.com/a/502168

%%fakesection Asymptote definitions
\usepackage{patch-asy}
\numberwithin{asy}{chapter}
\renewcommand{\theasy}{\thechapter\Alph{asy}}
\begin{asydef}
  import extras;
  size(6cm);
  usepackage("amsmath");
  usepackage("amssymb");
  defaultpen(fontsize(11pt));
  settings.tex = "latex";
  settings.outformat = "pdf";
\end{asydef}
\def\asydir{asy}

%%fakesection Bibliography
\usepackage[backend=biber,backref=true,style=alphabetic]{biblatex}
\DeclareLabelalphaTemplate{
  \labelelement{
    \field[final]{shorthand}
    \field{label}
    \field[strwidth=2,strside=left]{labelname}
  }
  \labelelement{
    \field[strwidth=2,strside=right]{year}
  }
}
\DeclareFieldFormat{labelalpha}{\textbf{\scriptsize #1}}
\addbibresource{references.bib}
\addbibresource{images.bib}
%% stylistic biblatex choices
\DefineBibliographyStrings{english}{%
  backrefpage  = {cited p.}, % for single page number
  backrefpages = {cited pp.} % for multiple page numbers
}
\DeclareFieldFormat{journaltitle}{\mkbibemph{#1},} % italic journal
% title with comma
\DeclareFieldFormat[inbook,thesis]{title}{\mkbibemph{#1}\addperiod} %
% italic title with period
\DeclareFieldFormat[article]{title}{#1} % title of journal article is
% printed as normal text
\DeclareFieldFormat[article]{volume}{\textbf{#1}\addcolon\space}
\renewcommand{\mkbibnamegiven}[1]{\textsc{#1}}
\renewcommand{\mkbibnamefamily}[1]{\textsc{#1}}
\renewcommand{\mkbibnameprefix}[1]{\textsc{#1}}
\renewcommand{\mkbibnamesuffix}[1]{\textsc{#1}}
\renewcommand{\finentrypunct}{}

%%fakesection Mini ToC
\usepackage[tight]{minitoc}
\mtcsetfont{parttoc}{chapter}{\sffamily\bfseries}
\mtcsetfont{parttoc}{section}{\footnotesize\rmfamily\upshape\mdseries}
\mtcsetfont{parttoc}{subsection}{\footnotesize\rmfamily\upshape\mdseries}
%\mtcsetdepth{parttoc}{1}
\setcounter{parttocdepth}{1}
\renewcommand*{\partheadstartvskip}{\vspace*{20em}}
\renewcommand*{\partheadendvskip}{}
%\noptcrule
\renewcommand\beforeparttoc{\noindent{\bfseries \Large Part \thepart: Contents}}
%\hspace{\fill}\rule{0.95\linewidth}{2pt}\hspace{\fill}
\doparttoc[n]

%%fakesection Misc haxx
\pdfstringdefDisableCommands{\def\Spec{\text{Spec }}\def\sigma{σ}}

% Additional packages
\usepackage{parskip}
\usepackage{soul}

% For quotes in chapter preambles
\renewcommand*{\dictumwidth}{.66\textwidth}
\renewcommand*{\dictumrule}{\vskip-1ex\hrulefill\par}
\renewcommand*{\dictumauthorformat}[1]{#1 \vskip1ex}

% For positioning figures
\usepackage{float}

% TikZ
\usetikzlibrary{decorations.pathreplacing}

% Censor
\usepackage{censor}
