\setchapterpreamble{\dictum[David S. Richeson, \textit{Euler's Gem:
  The Polyhedron Formula and the Birth of Topology}]{``If geometry is
dressed in a suit coat, topology dons jeans and a T-shirt.''}}
\chapter{Basic Topology of $\RR$}
\begin{definition}[Open sets]
  \deflabel{open-sets}
  \phantomsection
  \addcontentsline{toc}{section}{Open sets}
  A set $O \subseteq \RR$ is \vocab{open} if for all points $x \in O$
  there exists an $\epsilon$-neighborhood $V_\epsilon(x) \subseteq O$.
\end{definition}

\begin{example}
  Below are some examples and some non-examples of open sets.
  \begin{enumerate}
    \item The set $\RR$ is open.
    \item The empty set $\emptyset$ is open.
    \item The open interval $(a, b)$ is open.
    \item The intervals $(a, \infty)$ and $(-\infty, b)$ are open.
    \item The closed interval $[a, b]$ is \textit{not} open.
  \end{enumerate}
\end{example}

\begin{proposition}
  \prplabel{open-sets-via-union-and-intersection}
  1. The union of an arbitrary collection of open sets is open.

  2. The intersection of a finite collection of open sets is open.
\end{proposition}

\begin{proof}
  To prove part 1, we let $\set{O_\lambda : \lambda \in \Lambda}$ be
  a collection of open sets and let $O = \bigcup_{\lambda \in
  \Lambda} O_\lambda$. Let $x$ be an arbitrary element of $O$. In
  order to show that $O$ is open, \defref{open-sets} insists that we
  produce an $\epsilon$-neighborhood of $x$ completely contained in
  $O$. Since $x \in O$, we know that $x \in O_\lambda$ for some
  $\lambda \in \Lambda$. By definition of $O_\lambda$ being open,
  this implies that there is an $\epsilon$-neighborhood
  $V_\epsilon(x) \subseteq O_\lambda$. The fact that $O_\lambda
  \subseteq O$ allows us to conclude that $V_\epsilon(x) \subseteq
  O$. This completes the proof of part 1.

  For part 2, let $\set{O_1, O_2, \dots, O_N}$ be a finite collection
  of open sets. Now, if $x \in \bigcup_{k = 1}^{N} O_k$, then $x$ is
  an element of each of the open sets. By the definition of an open
  set, we know that, for each $1 \leq k \leq N$, there exists
  $V_{\epsilon_k}(x) \subseteq O_k$. We are in search of a single
  $\epsilon$-neighborhood of $x$ that is contained in every $O_k$, so
  the trick is to take the smallest one. Letting $\epsilon =
  \min(\epsilon_1, \epsilon_2, \dots, \epsilon_N)$, it follows that
  $V_\epsilon(x) \subseteq V_{\epsilon_k}(x)$ for all $k$, and hence
  $V_\epsilon(x) \subseteq \bigcap_{k = 1}^{N} O_k$, as desired.
\end{proof}

\begin{example}
  Some more examples of open sets.
  \begin{enumerate}
    \item All intervals $(a, b)$ are open, so by
      \prpref{open-sets-via-union-and-intersection}, $(a, b) \cup (c,
      d)$ is open, $(a, b) \cup (c, d) \cup (e, f)$ is open, $(a_1,
      b_1) \cup (a_2, b_2) \cup \dots$ is open, and any other union
      of open intervals is open.
    \item Likewise, $(a, b) \cap (c, d)$ is open.
  \end{enumerate}
\end{example}

\begin{theorem}[Characterization of open sets]
  \phantomsection
  \addcontentsline{toc}{section}{Characterization of open sets}
  Every open set is a countable union of disjoint open intervals.
\end{theorem}

\begin{proof}
  Let $O$ be an open set. Then by \defref{open-sets}, each $x \in
  O$ is contained inside some open interval $(x - \epsilon, x +
  \epsilon) \subseteq O$. There may be a larger open interval that
  contains $x$, though, so let $I_x \subseteq O$ be the largest open
  interval containing $x$. Formally, $I_x = (\alpha, \beta)$ where
  \[ \alpha = \inf\set{a : (a, x) \subseteq O} \qquad \text{and} \qquad
  \beta = \sup\set{b : (x, b) \subseteq O}. \]
  Note that, given any pair $x$ and $y$, either $I_x = I_y$ or $I_x
  \cap I_y = \emptyset$. The reason for this is that, if instead you
  had two intervals overlapping but not equal, then you could expand
  at least one of them, contradicting the fact that we had chosen the
  largest intervals.

  We claim that these $I_x$ intervals are the ones we are looking
  for. To see this, first note that since each $x \in O$ is in $I_x
  \subseteq O$, the union of all $I_x$ intervals does indeed equal to
  $O$. Moreover, there are only countably many such disjoint
  intervals due to \thmref{open-intervals-countable}. This completes the proof.
\end{proof}

\begin{definition}[Limit points]
  \phantomsection
  \addcontentsline{toc}{section}{Limit points}
  \deflabel{limit-points}
  A point $x$ is a \vocab{limit point} of a set $A$ if and only if
  every $\epsilon$-neighborhood $V_\epsilon(x)$ of $x$ intersects the
  set $A$ at some point other than $x$.
\end{definition}

\begin{theorem}
  \thmlabel{limit-points-alternative}
  A point $x$ is a limit point of a set $A$ if and only if there is a
  sequence $(a_n)$ contained in $A \setminus \set{x}$ such that $a_n \to x$.
\end{theorem}

\begin{proof}
  \phantom{.}

  $(\Rightarrow)$ Assume $x$ is a limit point of $A$. In order to
  produce a sequence
  $(a_n)$ converging to $x$, we are going to consider the particular
  $\epsilon$-neighborhoods
  obtained using $\epsilon = 1/n$. By \defref{limit-points}, every
  neighborhood of $x$ intersects
  $A$ in some point other than $x$. This means that, for each $n \in
  \NN$, we are justified
  in picking a point
  \[ a_n \in V_{1/n}(x) \cap A \]
  with the stipulation that $a_n \neq x$. It should not be too
  difficult to see why $a_n \to x$. Given an arbitrary $\epsilon >
  0$, choose $N$ such that $1/N \leq \epsilon$. It follows that
  $\abs{a_n - x} < \epsilon$ for all $n > N$.

  $(\Leftarrow)$ We assume $a_n \to x$ where $a_n \in A
  \setminus \set{x}$, and let $V_\epsilon(x)$ be an arbitrary
  $\epsilon$-neighborhood. The definition of convergence assures us
  that there exists a term $a_N$ in the sequence satisfying $a_N \in
  V_\epsilon(x)$, and the proof is complete.
\end{proof}

\begin{remark}
  The restriction that an $a_n \neq x$ in
  \thmref{limit-points-alternative} deserves a comment. Given a
  point $a \in A$, it is always the case that $a$ is the limit of a
  sequence in $A$ if we are allowed to consider the constant sequence
  $(a, a, a, \dots)$. There will be occasions where we will want to
  avoid this somewhat uninteresting situation, so it is important to
  have a vocabulary that can distinguish limit points of a set from
  \textit{isolated points}.
\end{remark}

\begin{definition}[Isolated points]
  \phantomsection
  \addcontentsline{toc}{section}{Isolated points}
  A point $a \in A$ is an \vocab{isolated point} of $A$ if it is not
  a limit point of $A$.
\end{definition}

\begin{definition}[Closed sets]
  \phantomsection
  \addcontentsline{toc}{section}{Closed sets}
  A set $F \subseteq \RR$ is \vocab{closed} if it contains its limit points.
\end{definition}

\begin{theorem}[Characterization of closed sets]
  \phantomsection
  \addcontentsline{toc}{section}{Characterization of closed sets}
  A set $F \subseteq \RR$ is closed if and only if every Cauchy
  sequence contained in $F$ has a limit that is also an element of $F$.
\end{theorem}
