\setchapterpreamble{\dictum[Tina Rapke, \textit{The View from Here: Confronting
  Analysis}]{``One of my favorite memories of studying was one night
    when analysis crept into my dreams. I woke up in a panicky cold
    sweat. In my dream I was being chased by some analysis monster. My
    only defense was to use the \textit{blancmange function} as a
    boomerang. I took it as a good sign at the time that analysis
concepts were finding their way into my subconscious.''}}
\chapter{Pathological Examples}
It has been said that one of the most important goals of learning
real analysis is to collect as many bizarre examples as you can, and
to keep them in your back pocket. From a practical standpoint they
will inform your conjectures and guide your proofs, but they will
also help to demonstrate why real analysis is such a great subject.

\section{An infinite field that cannot be ordered}
To say that a field $F$ cannot be ordered is to say that it possesses
no positive subset $P$ satisfying the order axiom
(\defref{ordered-fields}). A preliminary comment is that since every
ordered field is infinite, no finite field can be ordered.

An example of an \textit{infinite} field that cannot be ordered is
the field $\CC$ of complex numbers.

\begin{proof}
  To show that this is the case, assume that there does exist a
  positive subset $P$ of $\CC$. Consider the element $i \in \CC$,
  where $i$ is the imaginary unit. Since $i \neq 0$, there are two
  alternative possibilities. The first is $i \in P$ (i.e. $i$ is a
  positive element), in which case $i^2 = -1 \in P$ and $i^4 = 1 \in
  P$. Since $i^2$ and $i^4$ are additive inverses of each other, and
  since it is impossible for two additive inverses both to belong to
  $P$, we have obtained a contradiction. The other option is $-i \in
  P$, in which case $(-i)^2 = -1 \in P$ and $(-i)^4 = 1 \in P$. We
  get the same contradiction as before.
\end{proof}
