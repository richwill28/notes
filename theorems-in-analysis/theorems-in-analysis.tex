\documentclass[11pt,twoside=off,numbers=noenddot]{scrbook}

%%fakesection Load packages

\usepackage{lmodern}
\usepackage[pdfusetitle]{hyperref}
\ExplSyntaxOn
\sys_if_engine_luatex:T {
    \usepackage{luatex85}
}
\sys_if_engine_pdftex:T {
    \usepackage[T1]{fontenc}
}
\ExplSyntaxOff

% These are evan.sty
\usepackage{amsmath,amssymb,amsthm}
\usepackage{mathrsfs}
\usepackage[usenames,svgnames,dvipsnames]{xcolor}
\usepackage{textcomp}
\usepackage{enumerate}
\usepackage[textsize=scriptsize,shadow]{todonotes}
\usepackage{mathtools}
\usepackage{microtype}
\usepackage[normalem]{ulem}
\usepackage{stmaryrd}
\usepackage{wasysym}
\usepackage{multirow}
\usepackage{prerex}
\usepackage[nameinlink]{cleveref}

%%fakesection evan.sty macros
%Small commands
%% Napkin commands
\newcommand{\prototype}[1]{
    \emph{{\color{red} Prototypical example for this section:} #1} \par\medskip
}
\newenvironment{moral}{%
    \begin{mdframed}[linecolor=green!70!black]%
        \bfseries\color{green!50!black}}%
        {\end{mdframed}}

%%fakesection Links (hyperref loaded earlier implicitly)
\hypersetup{
    linkcolor={red!50!black},
    citecolor={green!50!black},
    urlcolor={blue!80!black},
    pdfkeywords={napkin,math},
    pdfsubject={web.evanchen.cc},
    colorlinks,
}

%%fakesection Commutative diagrams
\usepackage{tikz-cd}
\usetikzlibrary{arrows,arrows.meta}
% make a larger hook
% https://tex.stackexchange.com/questions/514451/how-to-define-a-new-hooked-arrow
\makeatletter
\pgfdeclarearrow{
    name=xGlyph,
    cache=false,
    bending mode=none,
    parameters={\tikzcd@glyph@len,\tikzcd@glyph@shorten},
    setup code={%
            \pgfarrowssettipend{\tikzcd@glyph@len\advance\pgf@x by\tikzcd@glyph@shorten}},
    defaults={
            glyph axis=axis_height,
            glyph length=+1.55ex,
            glyph shorten=+-0.1ex},
    drawing code={%
            \pgfpathrectangle{\pgfpoint{+0pt}{+-1.5ex}}{\pgfpoint{+\tikzcd@glyph@len}{+3ex}}%
            \pgfusepathqclip%
            \pgftransformxshift{+\tikzcd@glyph@len}%
            \pgftransformyshift{+-\tikzcd@glyph@axis}%
            \pgftext[right,base]{\tikzcd@glyph}}}
\makeatother
\tikzcdset{
arrow style=tikz,
diagrams={>={Latex}},
tikzcd left hook/.tip={xGlyph[glyph math command=supset, swap, glyph axis = 5.7pt]},
tikzcd right hook/.tip={xGlyph[glyph math command=supset, glyph axis = 5.7pt]},
surjective head arrow /.tip = {tikzcd to[sep=-1.5pt]tikzcd to},
surjective head/.style={
        -surjective head arrow
    }
}

%%fakesection Page layout
\usepackage[headsepline]{scrlayer-scrpage}
\renewcommand{\headfont}{}
\addtolength{\textheight}{3.14cm}
\setlength{\footskip}{0.5in}
\setlength{\headsep}{10pt}

\def\shortdate{\leavevmode\hbox{\the\year-\twodigits\month-\twodigits\day}}
\def\twodigits#1{\ifnum#1<10 0\fi\the#1}
\automark[chapter]{chapter}

\rohead{\footnotesize\thepage}
\rehead{\footnotesize \textbf{\sffamily Napkin}, by \emph{Evan Chen} (\napkinversion)}
\lehead{\footnotesize\thepage}
\lohead{\footnotesize \leftmark}
\chead{}
\rofoot{}
\refoot{}
\lefoot{}
\lofoot{}
%\cfoot{\pagemark}

%%fakesection Fancy section and chapter heads
\renewcommand*{\sectionformat}{\color{purple}\S\thesection\autodot\enskip}
\renewcommand*{\subsectionformat}{\color{purple}\S\thesubsection\autodot\enskip}
\newcommand{\problemhead}{A few harder problems to think about}
\renewcommand{\thesubsection}{\thesection.\roman{subsection}}

\addtokomafont{chapterprefix}{\raggedleft}
\RedeclareSectionCommand[beforeskip=0.5em]{chapter}
\renewcommand*{\chapterformat}{%
    \mbox{\scalebox{1.5}{\chapappifchapterprefix{\nobreakspace}}%
        \scalebox{2.718}{\color{purple}\thechapter\autodot}\enskip}}

\addtokomafont{partprefix}{\rmfamily}
\renewcommand*{\partformat}{\color{purple}\scalebox{2.5}{\thepart}}

%%fakesection Theorems
\usepackage{thmtools}
\usepackage[framemethod=TikZ]{mdframed}

\theoremstyle{definition}
\mdfdefinestyle{mdbluebox}{%
    linewidth=1pt,
    skipabove=12pt,
    innerbottommargin=9pt,
    skipbelow=2pt,
    nobreak=true,
    linecolor=blue,
    backgroundcolor=TealBlue!5,
}
\declaretheoremstyle[
    headfont=\sffamily\bfseries\color{MidnightBlue},
    mdframed={style=mdbluebox},
    headpunct={\\[3pt]},
    postheadspace={0pt}
]{thmbluebox}

\mdfdefinestyle{mdredbox}{%
    linewidth=0.5pt,
    skipabove=12pt,
    frametitleaboveskip=5pt,
    frametitlebelowskip=0pt,
    skipbelow=2pt,
    frametitlefont=\bfseries,
    innertopmargin=4pt,
    innerbottommargin=8pt,
    linecolor=RawSienna,
    backgroundcolor=Salmon!5,
}
\declaretheoremstyle[
    headfont=\bfseries\color{RawSienna},
    mdframed={style=mdredbox},
    headpunct={\\[3pt]},
    postheadspace={0pt},
]{thmredbox}

\declaretheorem[%
    style=thmbluebox,name=Theorem,numberwithin=section]{theorem}
\declaretheorem[style=thmbluebox,name=Lemma,sibling=theorem]{lemma}
\declaretheorem[style=thmbluebox,name=Proposition,sibling=theorem]{proposition}
\declaretheorem[style=thmbluebox,name=Corollary,sibling=theorem]{corollary}
\declaretheorem[style=thmredbox,name=Example,sibling=theorem]{example}

\mdfdefinestyle{mdgreenbox}{%
    skipabove=8pt,
    linewidth=2pt,
    rightline=false,
    leftline=true,
    topline=false,
    bottomline=false,
    linecolor=ForestGreen,
    backgroundcolor=ForestGreen!5,
}
\declaretheoremstyle[
    headfont=\bfseries\sffamily\color{ForestGreen!70!black},
    bodyfont=\normalfont,
    spaceabove=2pt,
    spacebelow=1pt,
    mdframed={style=mdgreenbox},
    headpunct={ --- },
]{thmgreenbox}
\declaretheoremstyle[
    headfont=\bfseries\sffamily\color{ForestGreen!70!black},
    bodyfont=\normalfont,
    spaceabove=2pt,
    spacebelow=1pt,
    mdframed={style=mdgreenbox},
    headpunct={},
]{thmgreenbox*}

\mdfdefinestyle{mdblackbox}{%
    skipabove=8pt,
    linewidth=3pt,
    rightline=false,
    leftline=true,
    topline=false,
    bottomline=false,
    linecolor=black,
    backgroundcolor=RedViolet!5!gray!5,
}
\declaretheoremstyle[
    headfont=\bfseries,
    bodyfont=\normalfont\small,
    spaceabove=0pt,
    spacebelow=0pt,
    mdframed={style=mdblackbox}
]{thmblackbox}

\declaretheorem[name=Question,sibling=theorem,style=thmblackbox]{ques}
\declaretheorem[name=Exercise,sibling=theorem,style=thmblackbox]{exercise}
\declaretheorem[name=Remark,sibling=theorem,style=thmgreenbox]{remark}
\declaretheorem[name=Remark,sibling=theorem,style=thmgreenbox*]{remark*}
\declaretheorem[name=Step,style=thmgreenbox]{step} % only used in Lebesgue int

\definecolor{darkmagenta}{rgb}{0.55, 0.0, 0.55}
\definecolor{patriarch}{rgb}{0.5, 0.0, 0.5}

\theoremstyle{definition}
\mdfdefinestyle{mdpurplebox}{%
    linewidth=1pt,
    skipabove=12pt,
    innerbottommargin=9pt,
    skipbelow=2pt,
    nobreak=true,
    linecolor=darkmagenta,
    backgroundcolor=patriarch!5,
}
\declaretheoremstyle[
    headfont=\sffamily\bfseries\color{darkmagenta},
    mdframed={style=mdpurplebox},
    headpunct={\\[3pt]},
    postheadspace={0pt}
]{thmpurplebox}
\declaretheorem[style=thmpurplebox,name=Definition,numberwithin=section]{definition}
\newtheorem{claim}[theorem]{Claim}
\newtheorem{fact}[theorem]{Fact}
\newtheorem{abuse}[theorem]{Abuse of Notation}

\newtheorem{problem}{Problem}[chapter]
\renewcommand{\theproblem}{\thechapter\Alph{problem}}
\newtheorem{sproblem}[problem]{Problem}
\newtheorem{dproblem}[problem]{Problem}
\renewcommand{\thesproblem}{\theproblem$^{\star}$}
\renewcommand{\thedproblem}{\theproblem$^{\dagger}$}
\newcommand{\listhack}{$\empty$\vspace{-2em}}

%%fakesection Answers
\usepackage{answers}
\Newassociation{hint}{answeritem}{tex/backmatter/all-hints}
\Newassociation{sol}{answeritem}{tex/backmatter/all-solns}
\renewcommand{\solutionextension}{out}
\renewenvironment{answeritem}[1]{\item[\bfseries #1.]}{}

%%fakesection Table of contents
% First add ToC to ToC
\makeatletter
\usepackage{etoolbox}
\pretocmd{\tableofcontents}{%
    \if@openright\cleardoublepage\else\clearpage\fi
    \pdfbookmark[0]{\contentsname}{toc}%
}{}{}%
\makeatother
\setcounter{tocdepth}{1}
\RedeclareSectionCommand[tocnumwidth=4.2em]{part}
\RedeclareSectionCommand[tocpagenumberwidth=2.2em,tocnumwidth=4.2em]{chapter}
\RedeclareSectionCommand[tocpagenumberwidth=2.2em,tocnumwidth=2.8em]{section}
% adjust tocpagenumberwidth manually for large page number: https://tex.stackexchange.com/a/502168

%%fakesection Asymptote definitions
\usepackage{patch-asy}
\numberwithin{asy}{chapter}
\renewcommand{\theasy}{\thechapter\Alph{asy}}
\begin{asydef}
    import extras;
    size(6cm);
    usepackage("amsmath");
    usepackage("amssymb");
    defaultpen(fontsize(11pt));
    settings.tex = "latex";
    settings.outformat = "pdf";
\end{asydef}
\def\asydir{asy}

%%fakesection Bibliography
\usepackage[backend=biber,backref=true,style=alphabetic]{biblatex}
\DeclareLabelalphaTemplate{
    \labelelement{
        \field[final]{shorthand}
        \field{label}
        \field[strwidth=2,strside=left]{labelname}
    }
    \labelelement{
        \field[strwidth=2,strside=right]{year}
    }
}
\DeclareFieldFormat{labelalpha}{\textbf{\scriptsize #1}}
\addbibresource{references.bib}
\addbibresource{images.bib}
%% stylistic biblatex choices
\DefineBibliographyStrings{english}{%
    backrefpage  = {cited p.}, % for single page number
    backrefpages = {cited pp.} % for multiple page numbers
}
\DeclareFieldFormat{journaltitle}{\mkbibemph{#1},} % italic journal title with comma
\DeclareFieldFormat[inbook,thesis]{title}{\mkbibemph{#1}\addperiod} % italic title with period
\DeclareFieldFormat[article]{title}{#1} % title of journal article is printed as normal text
\DeclareFieldFormat[article]{volume}{\textbf{#1}\addcolon\space}
\renewcommand{\mkbibnamegiven}[1]{\textsc{#1}}
\renewcommand{\mkbibnamefamily}[1]{\textsc{#1}}
\renewcommand{\mkbibnameprefix}[1]{\textsc{#1}}
\renewcommand{\mkbibnamesuffix}[1]{\textsc{#1}}
\renewcommand{\finentrypunct}{}

%%fakesection Mini ToC
\usepackage[tight]{minitoc}
\mtcsetfont{parttoc}{chapter}{\sffamily\bfseries}
\mtcsetfont{parttoc}{section}{\footnotesize\rmfamily\upshape\mdseries}
\mtcsetfont{parttoc}{subsection}{\footnotesize\rmfamily\upshape\mdseries}
%\mtcsetdepth{parttoc}{1}
\setcounter{parttocdepth}{1}
\renewcommand*{\partheadstartvskip}{\vspace*{20em}}
\renewcommand*{\partheadendvskip}{}
%\noptcrule
\renewcommand\beforeparttoc{\noindent{\bfseries \Large Part \thepart: Contents}}
%\hspace{\fill}\rule{0.95\linewidth}{2pt}\hspace{\fill}
\doparttoc[n]

%%fakesection Misc haxx
\pdfstringdefDisableCommands{\def\Spec{\text{Spec }}\def\sigma{σ}}
\usepackage{parskip}

% Style
\newenvironment{proofidea}{\begin{proof}[Proof Idea]}{\end{proof}}
\newcommand{\vocab}[1]{\textbf{\color{blue} #1}}

% Computer Science
\renewcommand{\L}{\textbf{L}}
\newcommand{\coNL}{\textbf{coNL}}
\newcommand{\NL}{\textbf{NL}}
\newcommand{\NSPACE}{\textbf{NSPACE}}
\newcommand{\PSPACE}{\textbf{PSPACE}}
\newcommand{\SPACE}{\textbf{SPACE}}
\newcommand{\TQBF}{\textbf{TQBF}}

% Mathematics
\newcommand{\CC}{\mathbb C}
\newcommand{\FF}{\mathbb F}
\newcommand{\NN}{\mathbb N}
\newcommand{\QQ}{\mathbb Q}
\newcommand{\RR}{\mathbb R}
\newcommand{\ZZ}{\mathbb Z}


\title{A Laundry List of Theorems in Analysis}
\author{Richard Willie}

\begin{document}

\maketitle

\chapter*{Preface}
These notes are a loose amalgamation of ideas, concepts, and explanations drawn from various sources, particularly the following books:
\begin{enumerate}
    \item Real Analysis: A Long-Form Mathematics Textbook by Jay Cummings \cite{cummings2019real}
    \item Understanding Analysis by Stephen Abbott \cite{abbott2015understanding}
    \item Introduction to Real Analysis by Bartle \& Sherbert \cite{bartle2011introduction}
    \item Mathematical Analysis by Tom M. Apostol \cite{apostol1974mathematical}
\end{enumerate}
They were originally meant for my own understanding and organization of thoughts, and as such, they may be unpolished, incomplete, or even occasionally incorrect.

I share them in the hope that they may serve as a useful reference, but they should not be treated as a primary source of learning. Readers are strongly encouraged to consult original texts and authoritative resources for a more rigorous and accurate treatment of the topics discussed.

Use these notes as a companion to your studies, not as a substitute for the depth and clarity provided by well-established literature.

\tableofcontents

\chapter{The Real Numbers}

\begin{definition}[Fields]
    A \vocab{field} is a nonempty set $\FF$, along with two binary operations, addition ($+$) and multiplication ($\cdot$), satisfying the following axioms.
    \begin{itemize}
        \item \textbf{Axiom 1 (Commutative Law).} If $a, b \in \FF$, then $a + b = b + a$ and $a \cdot b = b \cdot a$.
        \item \textbf{Axiom 2 (Distributive Law).} If $a, b \in \FF$, $a \cdot (b + c) = a \cdot b + a \cdot c$.
        \item \textbf{Axiom 3 (Associative Law).} If $a, b \in \FF$, then $(a + b) + c = a + (b + c)$ and $(a \cdot b) \cdot c = a \cdot (b \cdot c)$.
        \item \textbf{Axiom 4 (Identity Law).} There are special elements $\zero, \one \in \FF$, where $a + \zero = a$ and $a \cdot \one = a$ for all $a \in \FF$.
        \item \textbf{Axiom 5 (Inverse Law).} For each $a \in \FF$, there is an element $-a \in \FF$ such that $a + (-a) = \zero$. If $a \neq \zero$, then there is also an element $a^{-\one} \in \FF$ such that $a \times a^{-\one} = \one$.
    \end{itemize}
\end{definition}

\begin{example}
    Below are some examples and some non-examples of fields.
    \begin{itemize}
        \item The natural number $\NN$ do not form a field; they fail the first half of Axiom 4 and both halves of Axiom 5.
        \item The integers $\ZZ$ \textit{almost} form a field; they only fail the second half of Axiom 5.
        \item One can check that the rationals $\QQ$ form a field.
    \end{itemize}
\end{example}

\begin{definition}[Ordered Fields]
    An \vocab{ordered field} is a field $\FF$, along with the following additional axiom.

    \textbf{Axiom 6 (Order Axiom).} There is a nonempty subset $P \subseteq \FF$, called the \textit{positive elements}, such that
    \begin{enumerate}
        \item If $a, b \in P$, then $a + b \in P$ and $a \cdot b \in P$;
        \item If $a \in \FF$ and $a \neq \zero$, then either $a \in P$ or $-a \in P$, but not both.
    \end{enumerate}
\end{definition}

\begin{definition}[Inequalities]
    If $\FF$ is an ordered field and $a, b \in \FF$, then we say that ``$a < b$'' if $b - a \in P$. Likewise, $a \leq b$ means that either $a = b$ or $a < b$.

    We define ``$>$'' similarly.
\end{definition}

\begin{fact}[Properties of inequalities]
    For $a, b, c$ in an ordered field $\FF$:
    \begin{enumerate}
        \item If $a < b$, then $a + c < b + c$.
        \item Transitivity: If $a < b$ and $b < c$, then $a < c$.
        \item If $a < b$, then $ac < bc$ if $c > 0$, and $ac > bc$ if $c < 0$.
        \item If $a \neq 0$, then $a^2 > 0$.
    \end{enumerate}
\end{fact}

\begin{definition}[The absolute value function]
    If $\FF$ is an ordered field, define the \textit{absolute value} function $\abs{\cdot} : \FF \to \FF$ to be
    \begin{align*}
        \abs{x} = \begin{cases}
            x, & x \geq 0 \\
            -x, & x < 0.
        \end{cases}
    \end{align*}
\end{definition}

\begin{fact}[Properties of absolute values]
    \faclabel{properties-of-absolute-values}
    For $a, b$ in an ordered field $\FF$:
    \begin{enumerate}
        \item $\abs{a} \geq 0$, with equality if and only if $a = 0$.
        \item $\abs{a} = \abs{-a}$.
        \item $-\abs{a} \leq a \leq \abs{a}$.
        \item $\abs{a \cdot b} = \abs{a} \cdot \abs{b}$.
        \item $1/\abs{a} = \abs{1/a}$, if $a \neq 0$.
        \item $\abs{a/b} = \abs{a}/\abs{b}$, if $b \neq 0$.
        \item $\abs{a} \leq b$ if and only if $-b \leq a \leq b$.
    \end{enumerate}
\end{fact}

\begin{theorem}[The triangle inequality]
    If $\FF$ is an ordered field and if $x, y \in \FF$, then
    \[ \abs{x + y} \leq \abs{x} + \abs{y}. \]
\end{theorem}

\begin{proof}
    For $x, y \in \FF$, by \facref{properties-of-absolute-values} part 3 we have
    \[ -\abs{x} \leq x \leq \abs{x} \qquad \text{and} \qquad -\abs{y} \leq y \leq \abs{y}. \]
    Adding these two together gives
    \[ -(\abs{x} + \abs{y}) \leq x + y \leq \abs{x} + \abs{y}. \]
    And so, by \facref{properties-of-absolute-values} part 7,
    \[ \abs{x + y} \leq \abs{x} + \abs{y}. \]
\end{proof}

\begin{corollary}[The reverse triangle inequality]
    Assume that $\FF$ is an ordered field and $x, y \in \FF$. Then,
    \begin{align*}
        \abs{\abs{x} - \abs{y}} \leq \abs{x - y}.
    \end{align*}
\end{corollary}

\begin{proof}
    By \facref{properties-of-absolute-values} part 7, it suffices to show that
    \[ -\abs{x - y} \leq \abs{x} - \abs{y} \leq \abs{x - y}. \]
    We first show the right-hand one (that $\abs{x} - \abs{y} \leq \abs{x - y}$), and we will do so by applying an application of the triangle inequality. Let $a = x - y$ and $b = y$. Then by the triangle inequality,
    \[ \abs{a + b} \leq \abs{a} + \abs{b}. \]
    That is,
    \begin{align*}
        \abs{(x - y) + y} \leq \abs{x - y} + \abs{y} \\
        \abs{x} \leq \abs{x - y} + \abs{y}.
    \end{align*}
    Rearranging,
    \[ \abs{x} - \abs{y} \leq \abs{x - y}. \]
    We will use a similar approach to show that $-\abs{x - y} \leq \abs{x} - \abs{y}$. Let $c = y - x$ and $d = x$. By the triangle inequality,
    \[ \abs{c + d} \leq \abs{c} + \abs{d} \]
    That is,
    \begin{align*}
        \abs{(y - x) + x} \leq \abs{y - x} + \abs{x} \\
        \abs{y} \leq \abs{y - x} + \abs{x}.
    \end{align*}
    Reaaranging,
    \[ -\abs{y - x} \leq \abs{x} - \abs{y}, \]
    which by \facref{properties-of-absolute-values} part 2 implies that
    \[ -\abs{x - y} \leq \abs{x} - \abs{y}, \]
    as desired.
\end{proof}

\begin{corollary}[Triangle inequality corollaries]
    For both of the following, assume that $\FF$ is an ordered field and $x, y \in \FF$.
    \begin{enumerate}
        \item $\abs{x - y} \leq \abs{x} + \abs{y}$.
        \item $\abs{x + y} \geq \abs{\abs{x} - \abs{y}}$.
    \end{enumerate}
\end{corollary}

\begin{proof}
    \phantom{.}

    For part 1, replace $y$ with $-y$ in the triangle inequality.

    For part 2, replace $y$ with $-y$ in the reserve triangle inequality.
\end{proof}

\begin{theorem}[Cauchy-Schwarz inequality]
    If $a_1, \dots, a_n$ and $b_1, \dots, b_n$ are arbitrary real numbers, we have
    \[ \left(\sum_{i = 1}^{n} a_i b_i\right)^2 \leq \left(\sum_{i = 1}^{n} a_i^2\right) \left(\sum_{i  = 1}^{n} b_i^2\right). \]
\end{theorem}

\begin{proof}
    A sum of squares can never be negative. Hence we have
    \[ \sum_{i = 1}^{n} (a_i x + b_i)^2 \geq 0 \]
    for every $x \in \RR$, with equality if and only if each term is zero. This inequality can be written in the form
    \[ Ax^2 + 2Bx + C \geq 0 \]
    where
    \[ A = \sum_{i = 1}^{n} a_i^2, \qquad B = \sum_{i = 1}^n a_i b_i, \qquad C = \sum_{i = 1}^{n} b_i^2. \]
    If $A > 0$, put $x = -B/A$ to obtain $B^2 - AC \leq 0$, which is the desired inequality. Otherwise if $A = 0$, the rest of the proof is trivial.
\end{proof}

\begin{definition}[Upper and lower bounds]
    Let $S$ be an ordered field and $A \subseteq S$ be nonempty.
    \begin{enumerate}
        \item The set $A$ is \textit{bounded above} if there exists some $b \in S$ such that $x \leq b$ for all $x \in A$; in this case $b$ is called an \vocab{upper bound} of $A$.
        \item The \vocab{least upper bound} of $A$—if it exists—is some $b_0 \in S$ such that
        \begin{enumerate}
            \item $b_0$ is an upper bound of $A$, and
            \item if $b$ is any other upper bound of $A$, then $b_0 \leq b$. 
        \end{enumerate}
        Such a $b_0$ is also called the \vocab{supremum} of $A$ and is denoted $\sup(A)$.
        \item Likewise, the set $A$ is \textit{bounded below} if there exists some $b \in S$ such that $x \geq b$ for all $x \in A$; in this case, $b$ is called a \vocab{lower bound} of $A$.
        \item Again, like above, the \vocab{greatest lower bound} of $A$—if it exists—is some $b_0 \in S$ such that
        \begin{enumerate}
            \item $b_0$ is a lower bound of $A$, and
            \item if $b$ is any other lower bound of $A$, then $b_0 \geq b$.
        \end{enumerate}
        Such a $b_0$ is also called the \vocab{infimum} of $A$ and is denoted $\inf(A)$.
        \item If a set is both bounded above and bounded below, then it is simply \textit{bounded}.
    \end{enumerate}
\end{definition}

\begin{example}
    The propositions below are left without proof.
    \begin{itemize}
        \item The set $\NN = \set{1, 2, 3, \dots}$ has no upper bounds. Lower bounds on $\NN$ include $-17$, $1$, $0.123$, and $-\pi$. Note that $\sup(\NN)$ does not exist, but $\inf(\NN)$ = 1.
        \item The set $\QQ$ has no upper or lower bounds; consequently, $\sup(\QQ)$ and $\inf(\QQ)$ do not exist.
        \item $\sup(\set{\frac{1}{n} : n \in \NN}) = 1$; $\inf(\set{\frac{1}{n} : n \in \NN}) = 0$. Note that the supremum here is in the set, while the infimum is not in the set.
        \item $\sup(\set{\frac{n}{n + 1} : n \in \NN}) = 1$; $\inf(\set{\frac{n}{n + 1} : n \in \NN}) = \frac{1}{2}$. Note that the infimum here is in the set, while the supremum is not in the set.
        \item In $\QQ$ the set $\set{x \in \QQ : x^2 < 2}$ does not have a supremum. In $\RR$ it will—in fact, $\sup(\set{x \in \QQ : x^2 < 2}) = \sqrt{2}$.
    \end{itemize}
\end{example}

\begin{definition}[Completeness]
    Let $S$ be an ordered field. Then $S$ has the \vocab{least upper bound property} if given any nonempty $A \subseteq S$ where $A$ is bounded above, $A$ has a least supper bound in $S$. In other words, $\sup(A) \in S$ for every such $A$.

    Such a set $S$ is also called \vocab{complete}.
\end{definition}

\begin{theorem}[Existence and uniqueness of $\RR$]
    There exists a unique complete ordered field. We call this field \textit{the real numbers}, $\RR$.
\end{theorem}

We will only prove existence. For uniqueness, see \remref{uniqueness-of-r}.

\begin{proofidea}
    We have the ordered field of rational numbers, but they aren't complete—there are holes everywhere, and to get to $\RR$ we must fill in these gaps. There are several ways to do this. But the most common method, which we discuss now, uses \textit{Dedekind cuts}.

    Each real number is going to be a set; at this point we have the rationals constructed, so each real number is going to be represented by a set of rationals. The way you want to think about it is this: the real number $x$ is going to be represented by the set of all rational numbers strictly less than $x$. These sets are going to be called \textit{cuts}, and while we discuss them you can start convincing yourself that each real number will indeed correspond to a unique cut, and each cut corresponds to a unique real number. But first, let's formally discuss cuts.

    \begin{definition}[Dedekind cuts]
        A \vocab{cut} should be thought of as the set $(-\infty, b) \cap \QQ$. That is, all rational numbers up to a certain point. Formally, it is defined as any set $C_b$ satisfying the following three conditions.
        \begin{enumerate}
            \item $C_b \subseteq \QQ$, but $C_b \neq \emptyset$ and $C_b \neq \QQ$;
            \item If $p \in C_b$ and $q \notin C_b$, then $p < q$;
            \item If $p \in C_b$, then there exists some $q \in C_b$ where $p < q$.
        \end{enumerate}
    \end{definition}

    And then $\RR$ is defined as the set of all cuts.

    Although we have an intuitive picture of a cut, at the moment it is just a set satisfying the above properties. We are interested in putting an algebraic structure on this collection of cuts. Addition and order work quite smoothly. For a pair of cuts $C_a$ and $C_b$, define the following.
    \begin{itemize}
        \item $C_a + C_b := \set{p + q : p \in C_a, q \in C_b}$
        \item $C_a < C_b$ if and only if $C_a \subsetneq C_b$
    \end{itemize}

    One can verify that the addition of two cuts is still a cut, and that addition is commutative and associative, that the $\zero$ cut behaves as it should (in turn giving \textit{positive} and \textit{negative} cuts) and that additive inverses exist. The inequality has the property that, given any two cuts $C_a$ and $C_b$, exactly one of the following holds: $C_a < C_b$, $C_a = C_b$, or $C_b < C_a$.

    Defining multiplication is trickier, because if you simply multiply the two sets together you'll have massive negative numbers multiplying against each other, creating massive positive numbers. Intuitively, you want $C_a \cdot C_b = C_{ab}$. That is, the cut $(-\infty, a) \cap \QQ$ times the cut $(-\infty, b) \cap \QQ$ should equal the cut $(-\infty, a \cdot b) \cap \QQ$; but cuts aren't defined with such $a$ and $b$—they produce $a$ and $b$. One way around this is to first define multiplication for positive cuts. That is, if $C_a$ and $C_b$ are both positive (larger than the cut $\set{q \in \QQ : q < 0}$), then define
    \[ C_a \cdot C_b := \set{p \cdot q : p \in C_a, q \in C_b \text{ with } p, q \geq 0} \cup \set{q \in \QQ : q < 0}. \]
    With this you can then define a product of two negative cuts by setting the product equal to the product of the two corresponding cuts. To define multiplication between a positive and a negative cut, you know the product should be negative so one approach is to consider the multiplication when both are positive, and then translate the result to the corresponding negative cut. It's a hassle to write out, but that's the idea.

    One can then check that the product of two cuts is still a cut, that multiplicative inverses exist, and that multiplication is commutative and associative, as well as the remaining multiplicative/additive distributive and order properties. These would be quite annoying to work out in detail, but you can smile knowing someone carefully checked them.

    With our set built, and with the algebraic properties defined and their properties verified, we now know that $\RR$ is an ordered field. All that is left to show is that it is complete. This is done by showing that cuts satisfy the \textit{least upper bound property}. That is, if $\mathcal{C}$ is a collection of cuts which is bounded above (meaning there exists some cut $D$ such that $C \leq D$ for all $C \in \mathcal{C}$), then there exists a least upper bound (meaning there is a cut $M$ such that $M \leq D$ for all upper bounds $D$). The proof is this: Let $M = \bigcup_{C \in \mathcal{C}} C$, and show that
    \begin{enumerate}
        \item $M$ is a cut and therefore $M \in \RR$;
        \item $M$ is an upper bound of $\mathcal{C}$, but is smaller than all other upper bounds.
    \end{enumerate}
    With those, $\RR$ is a complete ordered field, and hence the real numbers are constructed.
\end{proofidea}

\begin{remark}
    \remlabel{uniqueness-of-r}
    The notion of uniqueness is only meaningful in relation to a specific axiomatization. That is, something can only be unique relative to a particular set of axioms. Now there are two common axiomatizations of the reals.

    One is a second order theory including the completeness axiom. In this particular axiomatization, there is a unique model of the second-order theory of the reals up to isomorphism. The core reason behind the uniqueness of complete ordered field is that any two such fields must contain an isomorphic copy of the rationals, and every element in each field cuts the rationals in the field into two parts, and the lower part has a least upper bound, and that different elements cut the rationals in different ways. This gives a one-to-one correspondence between the reals in one field to the reals in the other field. One can say that it is the rigidity and denseness of the rationals that is the key.

    However, the other common axiomatization of the reals is the theory of real closed fields. This axiomatization does \textit{not} yield a unique model.

    Reference: \url{https://math.stackexchange.com/a/2246530}
\end{remark}

\begin{proposition}[Axioms of $\RR$]
    The set $\RR$ has two binary operations, addition ($+$) and multiplication ($\cdot$), and is the unique set satisfying the following axioms.
    \begin{itemize}
        \item \textbf{Axiom 1 (Commutative Law).} If $a, b \in \RR$, then $a + b = b + a$ and $a \cdot b = b \cdot a$.
        \item \textbf{Axiom 2 (Distributive Law).} If $a, b \in \RR$, $a \cdot (b + c) = a \cdot b + a \cdot c$.
        \item \textbf{Axiom 3 (Associative Law).} If $a, b \in \RR$, then $(a + b) + c = a + (b + c)$ and $(a \cdot b) \cdot c = a \cdot (b \cdot c)$.
        \item \textbf{Axiom 4 (Identity Law).} There are special elements $\zero, \one \in \RR$, where $a + \zero = a$ and $a \cdot \one = a$ for all $a \in \RR$.
        \item \textbf{Axiom 5 (Inverse Law).} For each $a \in \RR$, there is an element $-a \in \RR$ such that $a + (-a) = \zero$. If $a \neq \zero$, then there is also an element $a^{-\one} \in \RR$ such that $a \times a^{-\one} = \one$.
        \item \textbf{Axiom 6 (Order Axiom).} There is a nonempty subset $P \subseteq \RR$, called the \textit{positive elements}, such that
        \begin{enumerate}
            \item If $a, b \in P$, then $a + b \in P$ and $a \cdot b \in P$;
            \item If $a \in \RR$ and $a \neq \zero$, then either $a \in P$ or $-a \in P$, but not both.
        \end{enumerate}
        \item \textbf{Axiom 7 (Completeness Axiom).} Given any nonempty $A \subseteq \RR$ where $A$ is bounded above, $A$ has a least upper bound. In other words, $\sup(A) \in \RR$ for every such $A$.
    \end{itemize}
\end{proposition}

\begin{proposition}[Suprema are unique]
    If the supremum or infimum of $A \subseteq \RR$ exists, then it is unique.
\end{proposition}

We will only prove that suprema are unique. The infima case is analogous.

\begin{proof}
    Assume for a contradiction that $\alpha$ and $\beta$ are distinct least upper bounds of $A$. In particular, both are upper bounds of $A$, while $\alpha \neq \beta$. One one hand, since $\alpha$ is a least upper bound and $\beta$ is an upper bound, we must have $\alpha \leq \beta$. On the other hand, since $\beta$ is a least upper bound and $\alpha$ is an upper bound, we must have $\beta \leq \alpha$. In summary,
    \[ \alpha \leq \beta \qquad \text{and} \qquad \beta \leq \alpha. \]
    This implies that $\alpha = \beta$, giving our contradiction.
\end{proof}

\begin{theorem}[Square roots exist]
    If $a \in \RR$ and $a \geq 0$, then $\sqrt{a} \in \RR$.
\end{theorem}

\begin{proofidea}
    One can show that $\sqrt{a} = \sup(\set{x \in \RR : x^2 < a})$, which is in $\RR$ by completeness.
\end{proofidea}

\begin{theorem}[Suprema analytically]
    \thmlabel{suprema-analytically}
    Let $A \subseteq \RR$. Then $\sup(A) = \alpha$ if and only if
    \begin{enumerate}
        \item $\alpha$ is an upper bound of $A$, and
        \item Given any $\epsilon > 0$, $\alpha - \epsilon$ is \textit{not} an upper bound of $A$. That is, there is some $x \in A$ for which $x > \alpha - \epsilon$.
    \end{enumerate}
    Likewise, $\inf(A) = \beta$ if and only if
    \begin{enumerate}
        \item $\beta$ is a lower bound of $A$, and
        \item Given any $\epsilon > 0$, $\beta + \epsilon$ is \textit{not} a lower bound of $A$. That is, there is some $x \in A$ for which $x < \beta + \epsilon$.
    \end{enumerate}
\end{theorem}

We will only prove the suprema case. The infima case is analogous.

\begin{proof}
    \phantom{.}

    $(\Rightarrow)$ First, assume that $\sup(A) = \alpha$. We aim to prove part 1 and 2. The first of these is immediate: Since $\sup(A) = \alpha$, $\alpha$ is the least upper bound of $A$, which of course also implies that it is an upper bound of $A$.

    Now we will show part 2. Let $\epsilon > 0$, Since $\alpha - \epsilon < \alpha$, we know that $\alpha - \epsilon$ is not an upper bound of $A$, because if so that would contradict $\alpha$ being the least upper bound of $A$. And so, since $\alpha - \epsilon$ is not an upper bound, there must be some $x$ who is greater than $\alpha - \epsilon$.

    $(\Leftarrow)$ Now assume part 1 and 2. We aim to prove that $\sup(A) = \alpha$. That is, we wish to show that $\alpha$ is an upper bound of $A$ (which is implied directly by part 1), and for any other upper bound $\beta$, we have $\alpha \leq \beta$. We have only the latter to prove. Assume that $\beta$ is some other upper bound of $A$, and assume for a contradiction that $\beta < \alpha$. Note that $0 < \alpha - \beta$. We will use $(\alpha - \beta)$ as our $\epsilon$, and then apply part 2 to contradict $\beta$ being an upper bound. 

    Now we will work it out formally. Let $\epsilon = \alpha - \beta$. Since $\epsilon > 0$, by part 2 there exists some $x \in A$ such that $x > \alpha - \epsilon = \alpha - (\alpha - \beta) = \beta$. But this is a contradiction, because we assumed that $\beta$ was an upper bound of $A$, and yet we found another element $x \in A$ that is larger than $\beta$.
\end{proof}

\begin{lemma}[The Archimedean principle]
    \lemlabel{archimedean-principle}
    If $a$ and $b$ are real numbers with $a > 0$, then there exists a natural number $n$ such that $na > b$.

    In particular, for any $\epsilon > 0$ there exists $n \in \NN$ such that $\frac{1}{n} < \epsilon$.
\end{lemma}

\begin{proof}
    We aim to show that $na > b$ for some $n \in \NN$; by dividing over the $a$, we aim to prove that there is some $n \in \NN$ such that $n > b/a$. Now, the number $b/a$ is just some real number that we know nothing about. In fact, let's just call it $x$. So, equivalently, we are trying to prove that given any real number $x$, there is some integer $n$ such that $n > x$.

    Assume for a contradiction that there is no integer larger than $x$. That is, assume that $x$ is an upper bound on the set $\NN$. Then $\NN$ is a subset of $\RR$ that is bounded above, and so by the completeness of $\RR$ we deduce that $\sup(\NN)$ exists. Call this supremum $\alpha$. Since $\alpha$ is the least upper bound of $\NN$, we know that $\alpha - 1$ is not an upper bound. That is, there exists some integer $m > \alpha - 1$. Adding 1 to each side,
    \[ m + 1 > \alpha. \]
    But this is a contradiction. If $\alpha$ is the supremum of $\NN$, then it is an upper bound on $\NN$. But we found $(m + 1) \in \NN$ which is larger than $\alpha$. This concludes the first statement in the principle.

    The second part follows directly from the first by letting $a = \epsilon$ and $b = 1$, and dividing over the $n$.
\end{proof}

\begin{example}
    Show that $\inf(\set{\frac{1}{n} : n \in \NN}) = 0$.
    \begin{proof}
        Let $A = \set{\frac{1}{n} : n \in \NN}$. We will use the analytic definition of suprema (\thmref{suprema-analytically}). We must then show that 0 is a lower bound of $A$ and that, for all $\epsilon > 0$, $0 + \epsilon$ is not a lower bound of $A$.

        The first of these is almost immediate: Since 1 and $n$ are positive for each $n \in \NN$, so is $1/n$. So $1/n > 0$, and thus 0 is indeed a lower bound for $A$.

        Working toward the second, let $\epsilon > 0$. Then by the Archimedean principle (\lemref{archimedean-principle}), there exists some $n \in \NN$ such that $\frac{1}{n} < \epsilon$. This element, $\frac{1}{n}$, is in $A$ and is less than $0 + \epsilon$. So $0 + \epsilon$ is not a lower bound of $A$.
    \end{proof}
\end{example}

\begin{example}
    Show that $\sup(\set{\frac{1}{n} : n \in \NN}) = 1$.
    \begin{proof}
        Let $A = \set{\frac{1}{n} : n \in \NN}$. We will use the analytic definition of suprema (\thmref{suprema-analytically}). We must then show that 1 is an upper bound of $A$ and that, for all $\epsilon > 0$, $1 - \epsilon$ is not an upper bound of $A$.

        For the first of these, note that since $n \geq 1$ for all $n \in \NN$, and by dividing over the $n$ we have that $1 \geq \frac{1}{n}$ for all $n \in \NN$. So 1 is indeed an upper bound for $A$.

        Working towards the second, let $\epsilon > 0$. We need to show that there is some $x \in A$ such that $1 - \epsilon < x$. But this is always accomplished by the number 1: Clearly $1 \in A$ and $1 - \epsilon < 1$.
    \end{proof}
\end{example}

\begin{definition}[Density]
    Suppose $A$ and $B$ are ordered field. Then $A$ is \vocab{dense} in $B$ if, for any $x, y \in B$, there exists $a \in A$ such that $x < a < y$.
\end{definition}

\begin{example}
    The propositions below are left without proof.
    \begin{itemize}
        \item $\QQ$ is dense in $\QQ$.
        \item $\RR \setminus \QQ$ is dense in $\QQ$.
        \item $\ZZ$ is \textit{not} dense in $\QQ$.
    \end{itemize}
\end{example}

\begin{lemma}
    \lemlabel{precursor-to-q-is-dense-in-r}
    Let $x, y \in \RR$. If $y - x > 1$, then there exists $z \in \ZZ$ such that $x < z < y$.
\end{lemma}

\begin{proof}
    First assume that $x$ and $y$ are at least 0, and consider the set
    \[ A = \set{n \in \NN_0 : n \leq x}. \]
    Since $x \geq 0$, this set is non-empty, and since it is a set of nonnegative integers which is bounded above by $x$, this set is finite. By induction on the size of $A$, we can show that $\max(A)$ exists and is an element of $A$. Call this maximum $M$. We claim $z := M + 1$ works.

    Note that since $M \in \NN_0$, also $z \in \NN_0$. Furthermore, since $z$ is larger than the largest element of $A$, $z$ is not in $A$, implying that $x < z$. Finally,
    \[ M \leq x \qquad \text{implies that} \qquad M + 1 \leq x + 1 \leq y. \]
    So $z < y$. In summary, we have shown that $x < z < y$, as desired.

    The cases where $x$ and $y$ are not at least 0 are similar. If both are negative, then by considering $-x$ and $-y$ the above argument gives an integer $z$ where $-y < z < -x$, showing that $-z$ works, since $x < -z < y$. If one is positive and one is negative, then 0 works.
\end{proof}

\begin{theorem}[$\QQ$ is dense in $\RR$]
    The rational numbers are dense in the real numbers.
\end{theorem}

\begin{proof}
    Pick any $x, y \in \RR$ where $x < y$. We need to show that there exists some $\frac{m}{n} \in \QQ$ (with $m, n \in \ZZ$) such that
    \[ x < \frac{m}{n} < y. \]
    First note that if $x < 0 < y$ then we are done, since $0 \in \QQ$. Furthermore, if we can show that the theorem holds for the case that $x$ and $y$ are positive, then it holds when they are negative ($0 < x < \frac{m}{n} < y$ implies $-y < \frac{-m}{n} < -x < 0$), so we may assume $x$ and $y$ are positive.

    Since $y - x > 0$, by the Archimedean principle (\lemref{archimedean-principle}) there exists some $n \in \NN$ such that $n (y - x) > 1$; i.e. $ny - nx > 1$. And so, by \lemref{precursor-to-q-is-dense-in-r}, there is some integer $m$ with
    \[ nx < m < ny. \]
    That is,
    \[ x < \frac{m}{n} < y, \]
    which concludes the proof.
\end{proof}

\begin{remark}
    The proof of \lemref{precursor-to-q-is-dense-in-r} also implies that, for any $x \in \RR$, there exists an integer $M$ such that $M \leq x \leq M + 1$. In particular, it implies that the floor and ceiling functions exist.
\end{remark}

\begin{definition}[Ceiling and floor functions]
    Let $x \in R$.
    \begin{itemize}
        \item The \textit{ceiling} of $x$, denoted $\ceil{x}$, is the integer $n$ such that $x \leq n < x + 1$.
        \item The \textit{floor} of $x$, denoted $\floor{x}$, is the integer $n$ such that $x  - 1 < n \leq x$.
    \end{itemize}
\end{definition}

\begin{definition}[Closed and open intervals]
    Define the \textit{closed interval} $[a, b]$ to be $\set{x \in \RR : a \leq x \leq b}$. Likewise the \textit{open interval} $(a, b)$ is defined to be $\set{x \in \RR : a < x < b}$, and half-open intervals and intervals to $\pm\infty$ are again exactly as you would expect.
\end{definition}

\begin{theorem}[Characterization of intervals]
    Let $S$ be a subset of $\RR$ that contains at least two points. If $S$ has the property such that
    \[ \text{if $x, y \in S$ and $x < y$, then $[x, y] \subseteq S$}, \tag{1} \]
    then $S$ is an interval.
\end{theorem}

\begin{proof}
    There are four cases to consider: (1) $S$ is bounded, (2) $S$ is bounded above but not below, (3) $S$ is bounded below but not above, and (4) $S$ is neither bounded above nor below.

    \textbf{Case (1).} Let $a = \inf(S)$ and $b = \sup(S)$. Then $S \subseteq [a, b]$ and we will show that $(a, b) \subseteq S$. If $a < z < b$, then $z$ is not a lower bound of $S$, so there exists $x \in S$ with $x < z$. Also, $z$ is not an upper bound of $S$, so there exists $y \in S$ with $z < y$. Therefore, $z \in [x, y]$, so property (1) implies that $z \in S$. Since $z$ is an arbitrary element of $(a, b)$, we conclude that $(a, b) \subseteq S$. Now if $a \in S$ and $b \in S$, then $S = [a, b]$. If $a \notin S$ and $b \notin S$, then $S = (a, b)$. The other possibilities lead to either $S = (a, b]$ or $S = [a, b)$.

    \textbf{Case (2).} Let $b = \sup(S)$. Then $S \subseteq (-\infty, b]$ and we will show that $(-\infty, b) \subseteq S$. If $z < b$, then $z$ is not an upper bound of $S$, so there exists $y \in S$ with $z < y$. Also, since $S$ is not bounded below, there exists $x \in S$ with $x < z$. By property 1, $z \in [x, y] \subseteq S$. Since $z$ is an arbitrary element of $(-\infty, b)$, we conclude that $(-\infty, b) \subseteq S$. Now if $b \in S$, then $S = (-\infty, b]$, and if $b \notin S$, then $S = (-\infty, b)$.

    \textbf{Case (3).} Let $a = \inf(S)$. Then $S \subseteq [a, \infty)$ and we will show that $(a, \infty) \subseteq S$. If $a < z$, then $z$ is not a lower bound of $S$, so there exists $x \in S$ with $x < z$. Also, since $S$ is not bounded above, there exists $y \in S$ with $z < y$. By property 1, $z \in [x, y] \subseteq S$. Since $z$ is an arbitrary element of $(a, \infty)$, we conclude that $(a, \infty) \subseteq S$. Now if $a \in S$, then $S = [a, \infty]$, and if $a \notin S$, then $S = (a, \infty)$.

    \textbf{Case (4).} We will show that $S = (-\infty, \infty)$. Pick any $x, y \in S$ with $x < y$, property 1 implies that $[x, y] \subseteq S$. Since $S$ is neither bounded above nor below, and the choice of $x$, $y$ is arbitrary, we conclude that $(-\infty, \infty) \subseteq S$. Also, every subset of $\RR$ is a subset of $(-\infty, \infty)$, thus $S = (-\infty, \infty)$.
\end{proof}

\begin{theorem}[The nested intervals property]
    \thmlabel{nested-intervals-property}
    For each $n \in \NN$, assume we are given a closed interval $I_n = [a_n, b_n]$. Also, assume that each $I_n$ contains $I_{n + 1}$. Then, the resulting nested sequence of closed intervals
    \begin{align*}
        I_1 \supseteq I_2 \supseteq I_3 \supseteq I_4 \supseteq \dots
    \end{align*}
    has a nonempty intersection. That is,
    \begin{align*}
        \bigcap_{n = 1}^{\infty} I_n \neq \emptyset.
    \end{align*}
\end{theorem}

\begin{proof}
    In order to show that $\bigcap_{n = 1}^{\infty} I_n \neq \emptyset$ is not empty, we are going to use the completeness axiom to produce a single real number $x$ satisfying $x \in I_n$ for every $n \in \NN$. Now, the completeness axiom is a statement about bounded sets, and the one we want to consider is the set
    \[ A = \set{a_n : n \in \NN} \]
    of left-hand endpoints of the interval. Because the intervals are nested, we see that every $b_n$ serves as an upper bound for $A$. Thus, we are justified in setting
    \[ x = \sup(A). \]
    Now, consider a particular $I_n = [a_n, b_b]$. Because $x$ is an upper bound for $A$, we have $a_n \leq x$. The fact that each $b_n$ is an upper bound for $A$ and that $x$ is the least upper bound implies $x \leq b_n$.

    Altogether then, we have $a_n \leq x \leq b_n$, which means $x \in I_n$ for every choice of $n \in \NN$. Hence, $x \in \bigcap_{n = 1}^{\infty} I_n$, and the intersection is not empty.
\end{proof}

\begin{remark}
    Note that the conclusion of \thmref{nested-intervals-property} need not hold if each $I_n$ is allowed to be an open interval.
\end{remark}

\chapter{Cardinality}

\begin{definition}[Cardinality]
    \deflabel{cardinality-bijection}
    Let $S$ and $T$ be sets. Then, $\abs{S} = \abs{T}$ if and only if there is a bijection from $S$ to $T$.
\end{definition}

\begin{definition}[Cardinality cont.]
    \deflabel{cardinality-injection}
    $\abs{S} \leq \abs{T}$ if and only if there is an injection from $S$ to $T$.
\end{definition}

\begin{remark}
    Our definitions above introduce two fundamental relations on cardinality, i.e. $\size{S} = \size{T}$ and $\size{S} \leq \size{T}$. We need to make sure that these relations have the mathematical properties we expect them to have. That is, we want $\size{S} = \size{T}$ to be an equivalence relation and $\size{S} \leq \size{T}$ to be a partial order.

    For the relation $\size{S} = \size{T}$, it's easy to show that it defines an equivalence relation:
    \begin{itemize}
        \item Reflexivity: Every set has a bijection with itself, i.e. $\size{S} = \size{S}$.
        \item Symmetry: If there is a bijection $f$ from $S$ to $T$, then $f^{-1}$ is a bijection from $T$ to $S$, and thus $\size{T} = \size{S}$.
        \item Transitivity: If there are bijections $f$ from $S$ to $T$ and $g$ from $T$ to $U$, then their composition $h = g \circ f$ is a bijection from $S$ to $U$. Hence, $\size{S} = \size{U}$.
    \end{itemize}

    For the relation $\size{S} \leq \size{T}$, we must establish that it's a partial order:
    \begin{itemize}
        \item Reflexivity: Every set has an injection to itself, i.e. $\size{S} \leq \size{S}$.
        \item Transitivity: If there exist injections $f$ from $S$ to $T$ and $g$ from $T$ to $U$, then their composition $h = g \circ f$ is an injection from $S$ to $U$. Hence, $\size{S} \leq \size{U}$.
    \end{itemize}

    The remaining property, antisymmetry, is where things get interesting, since it's not immediately obvious. Antisymmetry means that if $\size{S} \leq \size{T}$ and $\size{T} \leq \size{S}$, then $\size{S} = \size{T}$. Using our definition, this translates to: If there is an injection from $S$ to $T$ and an injection from $T$ to $S$, then there should be a bijection between $S$ and $T$. This is exactly what the Schröder-Bernstein theorem (\thmref{schröder-bernstein}) guarantees.
\end{remark}

\begin{theorem}[Schröder-Bernstein theorem]
    \thmlabel{schröder-bernstein}
    If there exist injections $f : A \to B$ and $g : B \to A$ between the sets $A$ and $B$, then there exists a bijection $h : A \to B$.

    In terms of the cardinality of the two sets, this implies that if $\size{A} \leq \size{B}$ and $\size{B} \leq \size{A}$, then $\size{A} = \size{B}$.
\end{theorem}

\begin{proof}
    The strategy is to partition $A$ and $B$ into components
    \[ A = X \cup X' \qquad \text{and} \qquad B = Y \cup Y' \]
    with $X \cap X' = \emptyset$ and $Y \cap Y' = \emptyset$, in such a way that $f$ is a surjection from $X$ to $Y$, and $g$ is a surjection from $Y'$ to $X'$. Achieving this would lead to a proof that there is a bijection $h$ from $A$ to $B$. Why? For all $x \in X'$, there exists a unique $y \in Y'$ satisfying $g(y) = x$. This means that there is a well-defined inverse function $g^{-1}(x) = y$ that maps $X'$ to $Y'$. Setting
    \begin{align*}
        h(x) = \begin{cases}
            f(x) & \text{if $x \in X$} \\
            g^{-1}(x) & \text{if $x \in X'$}
        \end{cases}
    \end{align*}
    gives the desired bijection from $A$ to $B$.

    Now, let $X_1 = A \setminus g(B)$ and inductively define a sequence of sets by letting $X_{n + 1} = g(f(X_n))$. We show that $\set{X_n : n \in \NN}$ is a pairwise disjoint collection of subsets of $A$, while $\set{f(X_n) : n \in \NN}$ is a similar collection in $B$.

    To see that the sets $X_1, X_2, X_3, \dots$ are pairwise disjoint, note that $X_1 \cap X_n = \emptyset$ for all $n \geq 2$ because $X_1 = A \setminus g(B)$ and $X_n \subseteq g(B)$. (Why? $f$ is a mapping from $A$ to $B$, so we have that $f(X_n) \subseteq B$, and thus $X_{n + 1} = g(f(X_n)) \subseteq g(B)$.) In the general case of $X_n \cap X_m$ where $1 < n < m$, note that if $x \in X_n \cap X_m$ then $f^{-1}(g^{-1}(x)) \in X_{n - 1} \cap X_{m - 1}$. Continuing in this way, we can show $X_1 \cap X_{m - n + 1}$ is not empty, which is a contradiction. Thus $X_n \cap X_m = \emptyset$. Just to be clear, the disjointness of the $X_n$ sets is not crucial to the overall proof, but it does help paint a clearer picture of how the sets $X$ and $X'$ are constructed.

    Let $X = \bigcup_{n = 1}^{\infty} X_n$ and $Y = \bigcup_{n = 1}^{\infty} f(X_n)$. We show that $f$ is a surjection from $X$ to $Y$. This is straightforward. Each $x \in X$ comes from some $X_n$ and so $f(x) \in f(X_n) \subseteq Y$. Likewise, each $y \in Y$ is an element of some $f(X_n)$ and thus $y = f(x)$ for some $x \in X_n \subseteq X$. Thus $f : A \to B$ is a surjection.

    Let $X' = A \setminus X$ and $Y' = B \setminus Y$. Let $y \in Y'$. Then $y \notin f(X_n)$ for all $n$ (by definition of $Y'$). We also conclude that $g(y) \notin X_{n + 1}$ for all $n$. (Why? Suppose if $g(y) \in X_{n + 1}$ for some $n$, then $g(y) \in g(f(X_n))$. That is, $g(y) = g(z)$ for some $z \in f(X_n)$, and since $g$ is injective, we have $y = z$ and thus $y \in f(X_n)$, which is a contradiction.) Clearly, $g(y) \notin X_1$ either (because $g(y) \subseteq g(B)$ and $X_1 = A \setminus g(B)$) and so $g$ is a mapping from $Y'$ to $X'$. To see that $g$ is a surjection from $Y'$ to $X'$, let $x \in X'$ be arbitrary. Because $X' \subseteq g(Y') \subseteq g(B)$, there exists $y \in B$ with $g(y) = x$. Could $y$ be an element of $Y$? No, because if $y \in Y$, $g(y)$ would be in $g(Y)$, and since $g(Y) \subseteq X$ (by definition of $Y$), this would mean $g(y) \in X$. But we're considering an $x \in X'$ with $g(y) = x$, so this is a contradiction as $X \cap X' = \emptyset$. Hence $y \in Y'$ and $g : Y' \to X'$ is a surjection.
\end{proof}

\begin{definition}[Cardinality cont.]
    \deflabel{cardinality-surjection}
    $\size{S} \geq \size{T}$ if and only if there is a surjection from $S$ to $T$.
\end{definition}

\begin{remark}
    Again, we must establish that $\size{S} \geq \size{T}$ defines a partial order. Reflexivity and transitivity are obvious. Now we will show antisymmetry. Suppose $\size{S} \geq \size{T}$ and $\size{T} \geq \size{S}$. Then, by definition, there are surjections from $S$ to $T$ and from $T$ to $S$. Using the axiom of choice, one can prove that there exists a surjection from $X$ to $Y$ if and only if there exists an injection from $Y$ to $X$. We conclude that $\size{T} \leq \size{S}$ and $\size{S} \leq \size{T}$, which implies $\size{S} = \size{T}$ by \thmref{schröder-bernstein}.
\end{remark}

\begin{theorem}[$\abs{\ZZ} = \abs{\NN}$]
    \thmlabel{cardinality-of-z-equal-n}
    There are as many integers as there are natural numbers.
\end{theorem}

\begin{proof}
    Define the function $f : \NN \to \ZZ$ with
    \begin{align*}
        f(n) = \begin{cases}
            (n - 1) / 2 & \text{if $n$ is odd} \\
            - n / 2 & \text{if $n$ is even}.
        \end{cases}
    \end{align*}
    Clearly, $f$ is a bijection. (See the following diagram for intuition.)
    \begin{align*}
        \NN : \phantom{.}\quad & \phantom{.}1 \quad \phantom{-}2 \quad \phantom{''''}3 \quad \phantom{-}4 \quad \phantom{''''}5 \quad \phantom{-}6 \quad \phantom{''''}7 \quad \cdots \\
        & \updownarrow \quad \phantom{.}\updownarrow \quad \phantom{.}\updownarrow \quad \phantom{.}\updownarrow \quad \phantom{.}\updownarrow \quad \phantom{.}\updownarrow \quad \phantom{.}\updownarrow \\
        \ZZ : \phantom{.}\quad & \phantom{.}0 \quad -1 \quad \phantom{''}1 \quad -2 \quad \phantom{''}2 \quad -3 \quad \phantom{''}3 \quad \cdots
    \end{align*}
    Hence, by \defref{cardinality-bijection}, $\size{\NN} = \size{\ZZ}$.
\end{proof}

\begin{remark}
    \thmref{cardinality-of-z-equal-n} also shows that two sets can have the same cardinality even if one is a proper subset of the other and the ``larger'' one even has infinitely many more elements than the ``smaller'' one. Make sure you take a moment to appreciate how remarkably, wonderfully weird this is.
\end{remark}

\begin{theorem}[$\abs{\QQ} = \abs{\NN}$]
    \thmlabel{cardinality-of-q-equal-n}
    There are as many rational numbers as there are natural numbers.
\end{theorem}

\begin{proof}
    Let $A_1 = \set{0}$ and for each $n \geq 2$, let $A_n$ be the set given by
    \[ A_n = \set{\pm\frac{p}{q} : \text{where $p, q \in \NN$ are in lowest terms with $p + q$ = n}}. \]
    The first few of these sets look like
    \[ A_1 = \set{0}, \qquad A_2 = \left\{\frac{1}{1}, \frac{-1}{1}\right\}, \qquad A_3 = \left\{\frac{1}{2}, \frac{-1}{2}, \frac{2}{1}, \frac{-2}{1}\right\}, \]
    \[ A_4 = \left\{\frac{1}{3}, \frac{-1}{3}, \frac{3}{1}, \frac{-3}{1}\right\}, \qquad \text{and} \qquad A_5 = \left\{\frac{1}{4}, \frac{-1}{4}, \frac{2}{3}, \frac{-2}{3}, \frac{3}{2}, \frac{-3}{2}, \frac{4}{1}, \frac{-4}{1}\right\}. \]
    The crucial observation is that each $A_n$ is finite and every rational number appears in exactly one of these sets. A bijection with $\NN$ is then achieved by consecutively listing the elements in each $A_n$.
    \begin{align*}
        \NN : \phantom{.}\quad & \;\;\, 1 \quad \phantom{-}2 \quad \phantom{''''}3 \quad \phantom{-}4 \quad \phantom{''''}5 \quad \phantom{-}6 \quad \phantom{''''}7 \quad \phantom{-}8 \quad \phantom{''''}9 \quad\:\,\, 10 \quad\,\, 11 \quad\,\, 12 \quad \cdots \\
        & \:\: \updownarrow \quad \phantom{.}\updownarrow \quad \phantom{.}\updownarrow \quad \phantom{.}\updownarrow \quad \phantom{.}\updownarrow \quad \phantom{.}\updownarrow \quad \phantom{.}\updownarrow \quad \phantom{.}\updownarrow \quad \phantom{.}\updownarrow \quad \phantom{.}\updownarrow \quad \phantom{.}\updownarrow \quad \phantom{.}\updownarrow \\
        \QQ : \phantom{.}\quad & \underbrace{0}_{A_1} \quad \underbrace{\frac{1}{1} \,\,\,\,\, -\frac{1}{1}}_{A_2} \quad \underbrace{\frac{1}{2} \,\,\,\,\, -\frac{1}{2} \quad\, \frac{2}{1} \,\,\,\,\, -\frac{2}{1}}_{A_3} \quad \underbrace{\frac{1}{3} \,\,\,\,\, -\frac{1}{3} \quad\, \frac{3}{1} \,\,\,\,\, -\frac{3}{1}}_{A_4} \quad \frac{1}{4} \:\:\:\:\: \cdots
    \end{align*}

    Admittedly, writing an explicit formula for this correspondence would be an awkward task, and attempting to do so is not the best use of time. What matters is that we see why every rational number appears in the correspondence exactly once. Given, say, $22/7$, we have that $22/7 \in A_{29}$. Because the set of elements in $A_1, \dots, A_{28}$ is finite, we can be confident that $22/7$ eventually gets included in the sequence. The fact that this line of reasoning applies to any rational number $p/q$ is our proof that the correspondence is surjective. To verify that it is injective, we observe that the sets $A_n$ were constructed to be disjoint so that no rational number appears twice. This completes the proof.
\end{proof}

\begin{theorem}[$\abs{\RR} > \abs{\NN}$]
    \thmlabel{cardinality-of-r-greater-than-n}
    There are more real numbers than natural numbers.
\end{theorem}

\begin{proof}
    Assume, for the sake of contradiction, that $\size{\RR} = \size{\NN}$, which implies that there is a bijection $f : \NN \to \RR$. What this suggests is that it is possible to enumerate the elements of $\RR$. If we let $x_1 = f(x_1)$, $x_2 = f(x_2)$, and so on, then the fact that $f$ is bijective (and thus surjective) means that we can write
    \[ \RR = \set{x_1, x_2, x_3, x_4, \dots} \tag{1} \]
    and be confident that every real number appears somewhere on the list. We will now use the nested intervals property (\thmref{nested-intervals-property}) to produce a real number that is not there.

    Let $I_1$ be a closed interval that does not contain $x_1$. Next, let $I_2$ be a closed interval, contained in $I_1$, which does not contain $x_2$. The existence of such $I_2$ is easy to verify. Certainly $I_1$ contains two smaller disjoint closed intervals, and $x_2$ can only be in one of these. In general, given an interval $I_n$, construct $I_{n + 1}$ to satisfy
    \begin{enumerate}
        \item $I_{n + 1} \subseteq I_n$ and
        \item $x_{n + 1} \notin I_{n + 1}$.
    \end{enumerate}
    We now consider the intersection $\bigcap_{n = 1}^{\infty} I_n$. If $x_k$ is some real number from the list in (1), then we have $x_k \notin I_k$, and it follows that
    \[ x_k \notin \bigcap_{n = 1}^{\infty} I_n. \]
    Now, we are assuming that the list in (1) contains every real number, and this leads to the conclusion that
    \[ \bigcap_{n = 1}^{\infty} I_n = \emptyset. \]
    However, the nested intervals property (\thmref{nested-intervals-property}) asserts that $\bigcap_{n = 1}^{\infty} I_n \neq \emptyset$. So there is at least one $x \in \bigcap_{n = 1}^{\infty} I_n$ that, consequently, cannot be on the list in (1). This contradiction means that such an enumeration of $\RR$ is impossible, and we conclude that $\size{\RR} \neq \size{\NN}$.

    We are not done yet. We still have to show that $\size{\RR} \geq \size{\NN}$. This step is straightforward. Define the function $g : \RR \to \NN$ with
    \begin{align*}
        g(x) = \begin{cases}
            n & \text{if $x = n$ for some $n \in \NN$} \\
            0 & \text{otherwise}.
        \end{cases}
    \end{align*}
    This function maps each real number that is a natural number to itself, and all other real numbers to 0. Since every natural number is the image of at least one real number under $g$, the function is surjective. Therefore, by \defref{cardinality-surjection}, we have $\size{\RR} \geq \size{\NN}$. Combining this with our previous result that $\size{\RR} \neq \size{\NN}$, we conclude that $\size{\RR} > \size{\NN}$.
\end{proof}

\begin{definition}[Countable and uncountable sets]
    \deflabel{countable}
    A set $S$ is \vocab{countable} if
    \begin{enumerate}
        \item its cardinality $\size{S}$ is less than or equal to $\size{\NN}$.
        \item there exists an injection from $S$ to $\NN$.
        \item $S$ is empty or there exists a surjection $\NN$ to $S$.
        \item there exists a bijection from $S$ to a subset of $\NN$.
        \item $S$ is either finite or \textit{countably infinite}.
    \end{enumerate}
    All of the definitions above are equivalent.

    A set $S$ is \vocab{countably infinite} if its cardinality $\size{S}$ is exactly $\NN$.

    A set $S$ is \vocab{uncountable} if it is not countable. That is, its cardinality $\size{S}$ is greater than $\size{\NN}$.
\end{definition}

\begin{corollary}[$\NN$ is countable]
    The set of natural numbers is countable.
\end{corollary}

\begin{corollary}[$\ZZ$ is countable]
    The set of integers is countable.
\end{corollary}

\begin{corollary}[$\QQ$ is countable]
    The set of rational numbers is countable.
\end{corollary}

\begin{corollary}[$\RR$ is uncountable]
    The set of real numbers is uncountable.
\end{corollary}

\begin{theorem}[Countable infinity is the smallest infinity]
    \thmlabel{countable-infinity-smallest}
    If $A \subseteq B$ and $B$ is countable, then $A$ is either countable or finite.
\end{theorem}

\begin{proof}
    $B$ is a countable set. Thus, there exists a bijection $f : \NN \to B$. Let $A \subseteq B$ be an infinite subset of $B$. We show that $A$ is countable. Let $n_1 \in \min\set{n \in \NN : f(n) \in A}$. As a start to a definition of $g : \NN \to A$, let $g(1) = f(n_1)$. Next let $n_2 = \min\set{n \in \NN : f(n) \in A \setminus \set{f(n_1)}}$ and let $g(2) = f(n_2)$. We inductively continue this process to produce a bijection $g$ from $\NN$ to $A$. In general, assume we have defined $g(k)$ for $k < m$, and let $g(m) = f(n_m)$ where $n_m = \min\set{n \in \NN : A \setminus \set{f(n_1), \dots, f(n_{k - 1})}}$.

    To show that $g$ is injective, observe that $m \neq m'$ implies $n_m \neq n_{m'}$ and it follows that $g(m) = f(n_m) \neq f(n_{m'}) = g(m')$ because $f$ is injective. To show that $g$ is surjective, let $a \in A$ be arbitrary. Because $f$ is surjective, $a = f(n')$ for some $n' \in \NN$. This means $n' \in \set{n : f(n) \in A}$ and as we inductively remove the minimal element, $n'$ must eventually be the minimum by at least the $(n' -1)$-th step.
\end{proof}

\begin{corollary}[$\size{\NN}$ is the smallest infinity]
    If $A \subseteq \NN$, then either $A$ is finite or $\size{A} = \size{\NN}$.
\end{corollary}

\begin{corollary}[Sizes of infinity]
    There are different sizes of infinity, with countable infinity being the smallest. Moreover, $\NN$, $\ZZ$ and $\QQ$ are countable while $\RR$ is uncountable.
\end{corollary}

\begin{theorem}[Countable union of countable sets is countable]
    A countable union of countable sets is countable. More precisely:
    \begin{enumerate}
        \item If $A_1, A_2, \dots, A_m$ are each countable sets, then $A_1 \cup A_2 \cup \dots \cup A_m$ is countable.
        \item If $A_n$ is a countable set for each $n \in \NN$, then the set $\bigcup_{n = 1}^{\infty} A_n$ is also countable.
    \end{enumerate}
\end{theorem}

\begin{proof}
    First, we prove part 1 for two countable sets, $A_1$ and $A_2$. Some technicalities can be avoided by first replacing $A_2$ with the set $B_2 = A_2 \setminus A_1 = \set{x \in A_2 : x \notin A_1}$. The point of this is that the union $A_1 \cup B_2$ is equal to $A_1 \cup A_2$ and the sets $A_1$ and $B_2$ are disjoint.

    Now, because $A_1$ is countable, there exists a bijection $f : \NN \to A_1$. If $B_2 = \emptyset$, then $A_1 \cup A_2 = A_1$ which we already know to be countable. If $B_2 = \set{b_1, b_2, \dots, b_m}$ has $m$ elements then define $h : \NN \to A_1 \cup B_2$ via
    \begin{align*}
        h(n) = \begin{cases}
            b_n & \text{if $n \leq m$} \\
            f(n - m) & \text{if $n > m$}.
        \end{cases}
    \end{align*}
    The fact that $h$ is bijective follows immediately from the same property of $f$. If $B_2$ is infinite, then by \thmref{countable-infinity-smallest} it is countable, and so there exists a bijection $g : \NN \to B_2$. In this case we define $h : \NN \to A_1 \cup B_2$ by
    \begin{align*}
        h(n) = \begin{cases}
            f((n + 1) / 2) & \text{if $n$ is odd} \\
            g(n / 2) & \text{if $n$ is even}.
        \end{cases}
    \end{align*}
    Again, the proof that $h$ is bijective is derived directly from the fact that $f$ and $g$ are both bijections. Graphically, the correspondence takes the form
    \begin{align*}
        \NN : \quad & \:\:\, 1 \quad\,\, 2 \quad\,\, 3 \quad\,\, 4 \quad\,\, 5 \quad\,\, 6 \quad\,\, \cdots \\
        & \,\, \updownarrow \quad\,\, \updownarrow \quad\,\, \updownarrow \quad\,\, \updownarrow \quad\,\, \updownarrow \quad\,\, \updownarrow \\
        A_1 \cup B_2 : \quad & \:\, a_1 \quad b_1 \,\,\,\,\, a_2 \quad b_2 \,\,\,\,\, a_3 \quad b_3 \:\:\,\,\,\, \cdots
    \end{align*}

    To prove the more general statement in part 1, we may use induction. We have just seen that the result holds for two countable sets. Now let's assume that the union of $m$ countable sets is countable, and show that the union of $m + 1$ countable sets is countable.

    Given $m + 1$ countable sets $A_1, A_2, \dots, A_{m + 1}$, we can write
    \[ A_1 \cup A_2 \cup \dots \cup A_{m + 1} = (A_1 \cup A_2 \cup \dots \cup A_m) \cup A_{m + 1}. \]
    Then $C_m = A_1 \cup \dots \cup A_m$ is countable by the induction hypothesis, and $C_m \cup A_{n + 1}$ is just the union of two countable sets which we know to be countable. This completes the proof for part 1.

    For part 2, induction cannot be used because we have an infinite number of sets. Instead, we show how arranging $\NN$ into the two-dimensional array
    \begin{align*}
        & \: 1 \quad\,\,\, \: 3 \quad\,\,\, \: 6 \quad\,\,\, 10 \quad\,\,\, 15 \:\:\,\,\, \cdots \\
        & \: 2 \quad\,\,\, \: 5 \quad\,\,\, \: 9 \quad\,\,\, 14 \quad\: \cdots \\
        & \: 4 \quad\,\,\, \: 8 \quad\,\,\, 13 \:\:\,\,\, \cdots \\
        & \: 7 \quad\,\,\, 12 \:\:\,\,\, \cdots \\
        & 11 \:\:\,\,\, \cdots \\
        & \,\, \vdots
    \end{align*}
    leads to a proof.

    Let's first consider the case where the sets $\set{A_n}$ are disjoint. In order to achieve bijection between $\NN$ and $\bigcup_{n = 1}^{\infty} A_n$, we first label the elements in each countable set $A_n$ as
    \[ A_n = \set{a_{n_1}, a_{n_2}, a_{n_3}, \dots}. \]
    Now arrange the elements of $\bigcup_{n = 1}^{\infty} A_n$ in an array similar to the one for $\NN$:
    \begin{align*}
        & A_1 \quad = \quad a_{11} \quad\,\, a_{12} \quad\,\, a_{13} \quad\,\, a_{14} \quad\,\, a_{15} \quad\,\, \cdots \\
        & A_2 \quad = \quad a_{21} \quad\,\, a_{22} \quad\,\, a_{23} \quad\,\, a_{24} \quad\,\, \cdots \\
        & A_3 \quad = \quad a_{31} \quad\,\, a_{32} \quad\,\, a_{33} \quad\,\, \cdots \\
        & A_4 \quad = \quad a_{41} \quad\,\, a_{42} \quad\,\, \cdots \\
        & A_5 \quad = \quad a_{51} \quad\,\, \cdots \\
        & \qquad\:\,\,\, \vdots
    \end{align*}
    This establishes a bijection $g : \NN \to \bigcup_{n = 1}^{\infty} A_n$ where $g(n)$ corresponds to the element $a_{jk}$ where $(j, k)$ is the row and column location of $n$ in the array for $\NN$.

    If the sets $\set{A_n}$ are not disjoint then our mapping may not be injective. In this case we could again replace $A_n$ with $B_n = A_n \setminus \set{A_1 \cup \dots \cup A_{n - 1}}$. Another approach is to use the previous argument to establish a bijection between $\bigcup_{n = 1}^{\infty} A_n$ and an infinite subset of $\NN$, and then appeal to \thmref{countable-infinity-smallest}. This completes the proof for part 2.
\end{proof}

\begin{theorem}[$\size{\QQ_+} = \size{\NN}$]
    There are as many positive rational numbers as there are natural numbers.
\end{theorem}

\begin{theorem}[$\RR \setminus \QQ$ is uncountable]
    There are uncountably many irrational numbers.
\end{theorem}

\begin{theorem}
    An uncountable collection of disjoint open intervals in $\RR$ cannot exist.
\end{theorem}

\begin{theorem}[$\size{(0, 1)} > \size{\NN}$]
    \thmlabel{cardinality-of-(0,1)-greater-than-n}
    There are more numbers in the open interval $(0, 1)$ than there are natural numbers.
\end{theorem}

\begin{theorem}[$\size{(0, 1)} = \size{\RR}$]
    \thmlabel{cardinality-of-(0,1)-equal-r}
    There are as many numbers in the open interval $(0, 1)$ as there are real numbers.
\end{theorem}

\begin{remark}
    \thmref{cardinality-of-(0,1)-equal-r} effectively establishes that \thmref{cardinality-of-r-greater-than-n} and \thmref{cardinality-of-(0,1)-greater-than-n} are equivalent.
\end{remark}

\begin{hypothesis}[The continuum hypothesis]
    There is no set whose cardinality is strictly between that of the naturals and the reals.
    \begin{align*}
        \size{\NN} < \size{S} < \size{\RR}.
    \end{align*}
\end{hypothesis}

\begin{theorem}[$\size{A} < \size{\power(A)}$]
    If $A$ is a set and $\power(A)$ is the power set of $A$, then
    \begin{align*}
        \size{A} < \size{\power(A)}.
    \end{align*}
\end{theorem}

\begin{corollary}[There exist infinitely many infinities]
    There exist infinitely many distinct infinite cardinalities.
\end{corollary}

\chapter{Sequences}

\begin{definition}
    A \vocab{sequence} of real numbers is a function $a : \NN \to \RR$.
\end{definition}

\begin{definition}
    A sequence $(a_n)$ is \vocab{bounded} if the range $\set{a_n : n \in \NN}$ is bounded. That is, if there exists a lower bound $L \in \RR$ and an upper bound $U \in \RR$ where
    \[ L \leq a_n \leq U \]
    for all $n$.
\end{definition}

\begin{proposition}
    A sequence $(a_n)$ is bounded if and only if there exists some $C \in \RR$ for which $\abs{a_n} \leq C$ for all $n$.
\end{proposition}

\begin{definition}
    A sequence $(a_n)$ \vocab{converges} to $a \in \RR$ if for all $\epsilon > 0$ there exists some $N \in \NN$ such that $\abs{a_n - a} < \epsilon$ for all $n > N$.

    When this happens, $a$ is called the \vocab{limit} of $a_n$.
\end{definition}

\begin{definition}
    Let $\epsilon > 0$. The \vocab{$\epsilon$-neighborhood} of a point $a$ is the interval
    \[ (a - \epsilon, a + \epsilon). \]
\end{definition}

\begin{definition}
    A sequence $(a_n)$ \textit{converges} to $a \in \RR$ if for all $\epsilon > 0$ there exists some $N \in \NN$ such that $a_n$ is in the $\epsilon$-neighborhood of $a$ for all $n > N$.
\end{definition}

\begin{definition}
    If a sequence $(a_n)$ does not converge, then it \vocab{diverges}.

    Divergence can come in three forms.
    \begin{enumerate}
        \item $(a_n)$ \textit{diverges to} $\infty$ (notation: $\lim_{n \to \infty} a_n = \infty$) if, for all $M > 0$, there exists some $N \in \NN$ such that $a_n > M$ for all $n > N$.
        \item $(a_n)$ \textit{diverges to} $-\infty$ (notation: $\lim_{n \to \infty} a_n = -\infty$) if, for all $M < 0$, there exists some $N \in \NN$ such that $a_n < M$ for all $n > N$.
        \item Otherwise, $(a_n)$'s limit \textit{does not exist}.
    \end{enumerate}
\end{definition}

\begin{proposition}[Limits are unique]
    A sequence cannot have more than one limit.
\end{proposition}

\begin{proposition}
    If $(a_n)$ is a convergent sequence, then $(a_n)$ is bounded.
\end{proposition}

\begin{theorem}[Sequence limit laws]
    Assume that $(a_n)$ and $(b_n)$ are convergent sequences of real numbers such that $a_n \to a$ and $b_n \to b$. Also assume that $c \in \RR$. Then,
    \begin{enumerate}
        \item $(a_n + b_n) \to a + b$.
        \item $(a_n - b_n) \to a - b$.
        \item $(a_n \cdot b_n) \to a \cdot b$.
        \item $(\frac{a_n}{b_n}) \to \frac{a}{b}$, provided each $b \neq 0$ and each $b_n \neq 0$.
        \item $(c \cdot a_n) \to c \cdot a$.
    \end{enumerate}
\end{theorem}

\begin{theorem}[Sequence squeeze theorem]
    Assume $a_n \leq x_n \leq b_n$ for all $n$. Furthermore, assume that
    \[ a_n \to L \qquad \text{and} \qquad b_n \to L. \]
    Then,
    \[ x_n \to L. \]
\end{theorem}

\begin{definition}
    A sequence $(a_n)$ is \vocab{monotone increasing} if $a_n \leq a_{n + 1}$ for all $n$. Likewise, a sequence $(a_n)$ is \vocab{monotone decreasing} if $a_n \geq a_{n + 1}$ for all $n$. If it is either monotone increasing or monotone decreasing, it is monotone.
\end{definition}

\begin{theorem}[The monotone convergence theorem]
    Suppose $(a_n)$ is monotone. Then $(a_n)$ converges if and only if it is bounded. Moreover,
    \begin{itemize}
        \item If $(a_n)$ is increasing, then either $(a_n)$ diverges to $\infty$ or
        \[ \lim_{n \to \infty} a_n = \sup(\set{a_n : n \in \NN}). \]
        \item If $(a_n)$ is decreasing, then either $(a_n)$ diverges to $-\infty$ or
        \[ \lim_{n \to \infty} a_n = \inf(\set{a_n : n \in \NN}). \]
    \end{itemize}
\end{theorem}

\begin{proposition}
    Suppose $S \subseteq \RR$ is bounded above. Then there exists a sequence $(a_n)$ where $a_n \in S$ for each $n$ and
    \[ \lim_{n \to \infty} a_n = \sup(S). \]
    Likewise, if $S$ is bounded below, then there exists a sequence $(b_n)$ where $b_n \in S$ for each $n$ and
    \[ \lim_{n \to \infty} b_n = \inf(S). \]
\end{proposition}

\begin{definition}
    Let $(a_n)$ be a sequence of real numbers and let
    \[ n_1 < n_2 < n_3 < \dots \]
    be an increasing sequence of integers. Then,
    \[ a_{n_1}, a_{n_2}, a_{n_3}, \dots \]
    is called a \vocab{subsequence} of $(a_n)$, and is denoted $(a_{n_k})$.
\end{definition}

\begin{proposition}
    A sequence $(a_n)$ converges to $a$ if and only if every subsequence of $(a_n)$ also converges to $a$.
\end{proposition}

\begin{corollary}
    If $(a_n)$ has a pair of subsequences converging to different limits, then $(a_n)$ diverges.
\end{corollary}

\begin{proposition}
    If a monotone sequence $(a_n)$ has a convergent subsequence, then $(a_n)$ converges too, and has the same limit.
\end{proposition}

\begin{lemma}
    Every sequence has a monotone subsequence.
\end{lemma}

\begin{theorem}[The Bolzano-Weierstrass theorem]
    Every bounded sequence has a convergent subsequence.
\end{theorem}

\begin{definition}
    A sequence $(a_n)$ is \vocab{Cauchy} if for all $\epsilon > 0$ there exists some $N \in \NN$ such that
    \[ \abs{a_m - a_n} < \epsilon \]
    for all $m, n > N$.
\end{definition}

\begin{lemma}
    If $(a_n)$ is Cauchy, then $(a_n)$ is bounded.
\end{lemma}

\begin{theorem}[Cauchy criterion for convergence]
    A sequence converges if and only if it is Cauchy.
\end{theorem}

% \chapter{Series}

% \chapter{The Topology of $\RR$}

% \chapter{Continuity}

% \chapter{Differentiation}

% \chapter{Integration}

% \appendix

% \chapter{Pathological Examples}

\printbibliography[nottype=image]

\end{document}
